\chapter{Literature Review}

Once upon a time engineers and researchers believed... In this area of research, they used the following methods... \cite{jct2010}

Write this section first as it will take you the longest. I suggest you start writing this as soon as you
have done your initial research at the beginning of your project. You can then return to it once you
have completed your work to edit and adjust it.

A literature review forms the theoretical basis of your project. You need to read a large number of
journal papers, sections in books, technical reports etc. relevant to your work at the start of project.
This will give you a good idea of the field of research.

When writing your review start of with the general concepts and move to the more specific aspects
explaining the necessary theory as you go. This section is NOT a copy and paste from others work or a
rewrite-but-change-one-word section. I suggest you read all your material, and then put it down and
write this section, referring back to the work only when you need to check something.

See your PCS textbook for more details on how to write a literature review.

If you include a figure or a table in your text please see the example in Fig. \ref{fig:model} as to how to caption it.
Please make sure that all text in your figures is readable and that you reference your figures if they are
from another source.

\begin{figure}[ht]
\centering
\includegraphics[width=0.7\textwidth]{model.png}
\caption{A block diagram illustrating the connections to the IRQ pin on the MCS08GT16A microcontroller (Please
note that your headings should be short descriptions of what is in the diagram not simply the figure title)}
\label{fig:model}
\end{figure}
\section{Gait sequence modelling and estimation}
\subsection{Quadrupede gait modelling}
\subsubsection{Periodicity}
\subsection{Quadrupede gait estimation}
\section{Case study: Inertia Measurement Unit}
\section{Hidden Markov Models}
Assumption of statistical model: Signal can be parametrised as a parametric random process and that the parametrers of the stochastic process can be determined/estimated in a precise, well-defined manner.\\
Observation is a probabilitistic function of the state - doubly embedded stochastic process.\\
Each state characterised by the probability distribution of observations, and transitions between states are characterised by a state transition matrix.
\subsubsection{First order Markov model - current and predecessor only considered}
\subsection{Transition Probability Matrix}
\subsection{Emission Probability Matrix}
\subsection{Initial distribution}
\subsection{Elements of an HMM}
\begin{enumerate}
	\item Number of states, N
	\item Number of distinct observation symbols per state, M
	%\item The state transition probability distribution A = \left\lbrace  a_{ij} \right\rbrace 
	%\item The observation symbol probability distribution in state j, B = \left\lbrace  b_{j}(k) \right\rbrace 
\item The initial state distribution, pi
\end{enumerate}
\subsection{Three fundamental problems for HMM design}
\begin{enumerate}
	\item Number of states, N
	\item Number of distinct observation symbols per state, M
%	\item The state transition probability distribution A = \left\lbrace  a_{ij} \right\rbrace 
%	\item The observation symbol probability distribution in state j, B = \left\lbrace  b_{j}(k) \right\rbrace 
%	\item The initial state distribution, pi.
\end{enumerate}
\subsection{Types of HMM}
\subsubsection{Ergodic model}
\subsubsection{Left-Right model or Bakis model}
\subsubsection{Evaluation of the probability of a sequence of observations}
\subsubsection{The determination of a best sequence states}
\subsubsection{The adjustment of model parameters to account for observed signal}
\section{k-Nearest Neighbour}
\section{Dimension reduction}
\subsection{Feature selection}
\section{Sufficiency of Training Data}
\section{Techniques to increase Training Data}
\subsection{Mirroring}