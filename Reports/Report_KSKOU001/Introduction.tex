\chapter{Introduction}

\begin{figure}
	\centering
	\includegraphics{Figures/BackPitch_vs_BackRoll}
%	\caption{accBackX_accBackY}
\end{figure}
\section{Background to the study}
A very brief background to your area of research. Start off with a general introduction to the area and
then narrow it down to your focus area. Used to set the scene \cite{smt2011}.
\section{Objectives of this study}
\subsection{Problems to be investigated}
Description of the main questions to be investigated in this study.
\subsection{Purpose of the study}
Give the significance of investigating these problems. It must be obvious why you are doing this study
and why it is relevant.

\section{Scope and Limitations}
Scope indicates to the reader what has and has not been included in the study. Limitations tell the
reader what factors influenced the study such as sample size, time etc. It is not a section for excuses as
to why your project may or may not have worked.

\section{Plan of development}
Here you tell the reader how your report has been organised and what is included in each
chapter.

{\bf I recommend that you write this section last. You can then tailor it to your report.}
