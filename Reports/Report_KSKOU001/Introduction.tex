\chapter{Introduction}

\section{Background to the study}
Bio-inspired robotics identifies useful mechanisms from nature to inform real-worl engineering systems. Similarly, the University of Cape Town (UCT) has set out to investigate how a cheetah uses its tail for stability during high acceleration, quick turns and sudden braking for sophisticated robots design. \\
To achieve this purpose, inertial measurement unit (IMU) data for a dog running, walking and trotting was acquired and labeled with the corresponding footfalls.\\
This dataset can therefore be used to perform gait sequence modelling and analysis using Hidden Markov Models (HMM), given its relatively small size. 						

\section{Objectives of this study}
The objective of the present work as stated in its outline is to "design, implement, and test (Hidden Markov Models) for estimating the gait sequence
from IMU data, so that specific models can be formulated and their
parameters estimated and interrogated." Thus, four main sub-objectives are formulated.
\begin{enumerate}
	\item Formulate HMM models to estimate the gait sequence of a dog from the labeled IMU measurements.
	\item Implement, test and evaluate the formulated models in order to estimate their parameters.
	\item Investigate how dimensionality reduction affects the performance of the designed models.
	\item Using the implemented models, perform useful gait pattern analysis such as motion recognition.
\end{enumerate}


\subsection{Problems to be investigated}
Description of the main questions to be investigated in this study.

The main questions to be answered are the following:
\begin{enumerate}
	\item How well can HMM model gait sequence dynamics using IMU data, in the abscence of enough training samples?
	\item Can dimensionality reduction cause an increase in performance of HMM models when there is not enough training data?
\end{enumerate}
\subsection{Purpose of the study}
Give the significance of investigating these problems. It must be obvious why you are doing this study
and why it is relevant.

\section{Scope and limitations}
The project was proposed with a very broad and flexible scope. In order to ensure completion of the project in the allocated time, its scope was therefore reduced to the essential.\\
More specifically, this project focuses on the design, the implementation and the performance evaluation of HMM models to identify the dog's footfalls from IMU measurements.\\
Moreover, particular attention was given to the design of dimensionality reduction techniques design and how they affect the performance of the HMM models when the training dataset is not large.\\
Furthermore, using the implemented models, motion type recognition was performed using the implemented HMM models.

The main constraint of this project was the relatively small size of the available IMU data.  
Besides, the capturing and calibration of the IMU data is not part of the scope of this project.

\section{Plan of development}
Here you tell the reader how your report has been organised and what is included in each
chapter.

{\bf I recommend that you write this section last. You can then tailor it to your report.}
