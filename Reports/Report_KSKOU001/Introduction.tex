\chapter{Introduction} \label{chap:intro}

Human motion analysis is an important research topic motivated by medical, artistic or scientific purposes \cite{towa2009}. Among others reasons, it is investigated for human activity monitoring, fall detection, gesture recognition, balance control evaluation, and abnormal behaviour detection \cite{cont2013}. To carry out these studies, micro-electro-mechanical inertial measurement units (IMUs) are increasingly used to capture motion data. Primarily, because they are lightweight and cheap \cite{ches2012} \cite{towa2009} \cite{cont2013}.\\
Gait analysis requires the identification of latent gait states \cite{cont2013}. A gait state being a particular stance in a given gait activity.
For this reason, it is extensively used in biologically inspired robot design. More specifically, understanding animals gait mechanism is very invaluable in bio-robotics.\\
Although, human gait analysis has been successfully performed using IMU, the threshold-based algorithms, %%TODO: ref 30 in "COnt'
and the fuzzy logic methods %%TODO: ref 8 in "COnt' 
used, lack robustness \cite{cont2013}. Thus, a more robust probabilistic technique is investigated in the present work. In this final year undergraduate project, we use Hidden Markov Model (HMM) to perform gait sequence modelling and estimation with IMU measurements of a moving dog.


\section{Background to the study}
Bio-inspired robotics identifies useful mechanisms in nature to inform real-world engineering systems. In this vein, the University of Cape Town (UCT) set out to investigate how a cheetah uses its tail for stability during high acceleration, quick turns and sudden braking for sophisticated robots design. \\
To achieve this purpose, inertial measurement unit (IMU) data for a dog running, walking and trotting was acquired and labelled with the corresponding footfalls.\\
This dataset can, therefore, be used to perform gait sequence modelling and analysis using Hidden Markov Models (HMM), given its relatively small size. 	%%TODO: add to gain more insight into a dog's gait to mimic a cheetah					

\section{Objectives of this study}
The objective of the present work as stated in its outline is to "design, implement, and test (Hidden Markov Models) for estimating the gait sequence
from IMU data, so that, specific models can be formulated and their
parameters estimated and interrogated." Thus, four main sub-objectives are formulated.
\begin{itemize}
	\item Formulate HMM models to estimate the gait sequence of a dog from the labelled IMU measurements.
	\item Implement, test and evaluate the formulated models in order to estimate their parameters.
	\item Investigate how dimensionality reduction affects the performance of the designed models.
	\item Using the implemented models, perform useful gait pattern analysis such as motion recognition.
\end{itemize}

To achieve the above objectives the hypotheses have been formulated:
\begin{itemize}
	\item HMMs can successfully model gait sequence dynamics using IMU data, in the absence of huge training data.
	\item Data aggregation and mirroring techniques can be used to overcome training data size limitations.
	\item Dimensionality reduction can increase the performance of an HMM, in the absence of a large training set.
	\item HMMs can be used to successfully perform gait activity recognition.
\end{itemize}


\section{Scope and limitations}
The project was proposed with a very broad scope. In order to ensure completion of the project in the allocated time, its scope was therefore reduced to the essential aspects.\\
Thus, the scope includes the continuous review of literature relevant to the completion of the project.
Moreover, the project focuses on the design, the implementation, and the performance evaluation of HMM models to identify the dog's footfalls from IMU measurements. This involves learning the use of programming tools, necessary to carry out the implementation.\\
Moreover, particular attention was paid to the design of dimensionality reduction techniques and how they affect the performance of the HMM models when the training dataset is not large.\\
Furthermore, motion type recognition was performed using the implemented HMM models.
 
The capturing and the calibration of the IMU data is not part of the scope of this project. Furthermore, new implementation of standard algorithms falls outside of the scope of this project. 

The main constraint of this project was the relatively small size of the available IMU data. In addition, the timespan of the project was just over 12 weeks.
\section{Plan of development}
The present report is organised as follows:
\begin{itemize}
	\item \textit{Chapter 1} - \textbf{Introduction:} This chapter sets out the scene of the project\ref{chap:intro}. 
	\item \textit{Chapter 2} - \textbf{Literature review and basic theory:} This section presents the basic theory on HMMs and dimensionality reduction based on the information found in literature\ref{chap:lit}.
	\item \textit{Chapter 3} - \textbf{HMM and dimensionality reduction methods design:} This is the design section where the HMM models are discussed. The practical implementation is also discussed here.\ref{chap:design}.
	\item \textit{Chapter 4}  \textbf{Design implementation:} In this chapter, the practical implementation is briefly discussed.
	\item \textit{Chapter 5} - \textbf{Experiments and results:} In this chapter, experiments are performed, results are presented, discussed and conclusions are drawn in an isolated manner.\ref{sec:results}.
	\item \textit{Chapter 6} - \textbf{General discussions:} This section ties the outcomes of the experiments togther in a broader sense according to the objectives of the project.
	\item \textit{Chapter 7} - \textbf{General conclusions:} The projects hypotheses are evaluated in light of the results and the discussions.
	\item \textit{Chapter 8} - \textbf{Recommendations for future work:} In the final chapter, recommendations are made for possible improvements and future work.
\end{itemize}
