% M. S. Tsoeu (2011), University of Cape Town <mohohlo.tsoeu@uct.ac.za>

% This is a project report templace document created for EEE4022FS students at the University of Cape Town.
%
% This file should be is processed with ``pdflatex`` and might need a few modifications if a different processor is chosen.


\documentclass[a4paper,12pt]{report}

%Include packages you need to use here

\usepackage[top = 1in, bottom = 1in, left = 1in, right = 1in]{geometry}
\usepackage{graphicx}
\usepackage{fancyhdr}
\usepackage{amsmath, amsthm, amssymb}
\usepackage{lastpage}
\usepackage{subfigure}
\usepackage{lscape}
\usepackage{hyphenat}
\usepackage{setspace}
\usepackage{hyperref}
%\usepackage[inline]{enumitem}   

%\usepackage{longtable}
\usepackage{multirow}
\usepackage{amsmath}
\usepackage{listings}
\usepackage{color}
%\usepackage{subcaption}
%\usepackage[export]{wrapfigure}

\definecolor{codegreen}{rgb}{0,0.6,0}
\definecolor{codegray}{rgb}{0.5,0.5,0.5}
\definecolor{codepurple}{rgb}{0.58,0,0.82}
\definecolor{backcolour}{rgb}{0.95,0.95,0.92}

\lstdefinestyle{mystyle}{
	backgroundcolor=\color{backcolour},   
	commentstyle=\color{codegreen},
	keywordstyle=\color{magenta},
	numberstyle=\tiny\color{codegray},
	stringstyle=\color{codepurple},
	basicstyle=\footnotesize,
	breakatwhitespace=false,         
	breaklines=true,                 
	captionpos=b,                    
	keepspaces=true,                 
	numbers=left,                    
	numbersep=5pt,                  
	showspaces=false,                
	showstringspaces=false,
	showtabs=false,                  
	tabsize=2
}

\lstset{style=mystyle}

% Include page formatting here. 
\parskip = 6mm
\parindent = 0mm
\renewcommand{\headrulewidth}{0pt}
\rhead[]{\thesection}
\lhead[\thechapter]{}


\begin{document}
 
% This section formats the title page of the Report.
\thispagestyle{empty}
{\Huge \begin{center}
% Modify the line below to insert your title.
Gait sequence modelling and estimation
\hrule 
% Modify the line below to insert your subtitle.
{\Large using Hidden Markov Models}
\end{center}}

\vskip 5mm
\begin{center}
\- \- \- \- \- \- \- \- \- \-\includegraphics[scale = 0.3]{uctLogo.png}
\end{center}

\vskip 5mm
\begin{center}
Presented by:\\
Kouame Hermann Kouassi		% Insert your name here
\end{center}

\vskip 10mm
\begin{center}
Prepared for:\\
Fred Nicolls\\ 		% Insert your supervisor's name here.
Dept. of Electrical and Electronics Engineering\\University of Cape Town
\end{center}


\vskip 10mm
\begin{center}
Submitted to the Department of Electrical Engineering at the University of Cape Town in partial
fulfilment of the academic requirements for a Bachelor of Science degree in Electrical and Computer Engineering

\end{center}


\vskip 5mm
\begin{center}{\bf \today}
\end{center}

\newpage
\thispagestyle{empty}
\mbox{}
\newpage

\onehalfspacing
\nohyphens{
\thispagestyle{empty}
\vskip 40mm


% Please leave the declaration as it is (Standard UCT declaration).
{\Large Declaration}\\
\hrule

\vskip 10mm
\begin{enumerate}
\item I know that plagiarism is wrong. Plagiarism is to use another's work and pretend that it is one's
own.
\item I have used the IEEE convention for citation and referencing. Each contribution to, and quotation in,
this report from the work(s) of other people has been attributed, and has been cited and
referenced.
\item This report is my own work.
\item I have not allowed, and will not allow, anyone to copy my work with the intention of passing it off
as their own work or part thereof.
\end{enumerate}
\vskip 10mm
Signature:\ldots\ldots\ldots\ldots\ldots\ldots\ldots\ldots\ldots 
\\Kouame H. Kouassi 		% Chante this line to your name.
\vskip10mm
Date:\ldots\ldots\ldots\ldots\ldots\ldots\ldots\ldots\ldots\ldots .


\fancyfoot[C]{\thepage}
\pagestyle{plain}
\newpage
\pagenumbering{roman}
{\Large Acknowledgments}\\
\hrule

\newpage

{\Large Abstract}\\
\hrule

% Place your abstract here.
Hidden Markov Models (HMMs) are doubly embedded stochastic processes with a rich underlying statistical structure. 
In the present work, gait sequence modelling and estimation was performed with HMMs using inertial measurement unit (IMU) data. More specifically, from IMU measurements, a dog's footfalls were correctly identified with up to 95\% precision. The continuous-valued IMU measurements were modelled with Gaussian mixtures. The originality of this project lies in the following three reasons. \\\\
Firstly, the similar work found in literature only uses 6-dimensional accelerometer and gyroscope measurements \cite{ches2012} \cite{cont2013}. The gait estimation was performed using 18-dimensional data. Thus, feature extraction and feature subset selection techniques were used for dimensionality reduction. The experiments demonstrated that up to 78\% performance increase is obtainable by using dimensionality reduction when, the training data is not large.\\\\
Secondly, the available IMU dataset was very small in size. Two different methods were employed to overcome this limitation. \\In the first method,
the reverse gait measurements were appended to the original data to increase the training data. In addition, the IMU measurements of the dog running, walking, and trotting were aggregated. 
Subsequently, it was possible to build more than 90\% accurate HMM model with 18-dimensional IMU measurements with low variance error.\\\\
Thirdly, the HMM gait estimation model was designed by building separate models for the front and the back legs of the dog. It, therefore, follows that, the gait estimation algorithm developed in this projet is applicable to both quadrupeds and bipeds.\\\\
Finally, the designed HMM models were successfully used to perform motion type recognition. Overall, this undergraduate final year project may be deemed successful.
\newpage
\tableofcontents

\newpage
\listoffigures

\newpage
\listoftables

% Page formatting, to place section titles as headers of odd pages and Chapter titles as headers of even pages.
\newpage
\fancyhead[RE,LO]{}
\fancyhead[LE]{\leftmark}
\fancyhead[RO]{\rightmark}
\pagestyle{fancy}

\pagenumbering{arabic}

% THe files included below are .tex files containing the respective chapters these are already created in this package and you can add to or modify them.
\chapter{Introduction}

\begin{figure}
	\centering
	\includegraphics{Figures/BackPitch_vs_BackRoll}
%	\caption{accBackX_accBackY}
\end{figure}
\section{Background to the study}
A very brief background to your area of research. Start off with a general introduction to the area and
then narrow it down to your focus area. Used to set the scene \cite{smt2011}.

Bio-inspired robotics uses nature to inform real-world engineering
systems. Research has been conducted at UCT to investigate the
manner in which a cheetah uses its tail for stability during high
acceleration, quick turns and sudden braking, with an aim to
incorporating identified mechanisms into sophisticated robot designs.
One way to acquire useful data is to strap an inertial measurement unit
(IMU) to an animal, and log the sensor data while certain actions are
being performed. We currently have such a dataset of a dog moving,
along with corresponding video data.						

\section{Objectives of this study}
The objective of this project is to design, implement, and test Hidden Markov Models (HMM) for estimating gait sequence from Inertia Measurement Unit (IMU) data.

so that specific models can be formulated and their
parameters estimated and interrogated. The project can be extended to
include any other useful analysis of gait patterns from similar sensor
measurements


1 - formulate model
2 - estimate its parameters
3 - Interrogate its parameters
4 - Useful analysis of gait patterns from IMU measurements
\subsection{Problems to be investigated}
Description of the main questions to be investigated in this study.

The main questions to be answered are the following:
\begin{enumerate}
	\item How well can HMM model gait sequence dynamics using IMU data, in the abscence of enough training samples?
	\item Can dimensionality reduction cause an increase in performance of HMM models when there is not enough training data?
\end{enumerate}
\subsection{Purpose of the study}
Give the significance of investigating these problems. It must be obvious why you are doing this study
and why it is relevant.

\section{Scope and Limitations}
Scope indicates to the reader what has and has not been included in the study. Limitations tell the
reader what factors influenced the study such as sample size, time etc. It is not a section for excuses as
to why your project may or may not have worked.

1 - Does not include data collection
2 - Focus on design of HMM only
3 - Focus on analysis of the model
4 - Focus on impact of dimensionality reduction

\section{Plan of development}
Here you tell the reader how your report has been organised and what is included in each
chapter.

{\bf I recommend that you write this section last. You can then tailor it to your report.}

\chapter{Literature Review}
\section{Gait sequence modelling and estimation}
\subsection{Quadrupede gait modelling}
\subsubsection{Periodicity}
\subsection{Quadrupede gait estimation}
\section{Case study: Inertia Measurement Unit}
\section{Hidden Markov Models}
Hidden Markov Models (HMMs) are doubly embedded stochastic processes with a rich underlying statistical structure. Introduced at the end of the 1960s by Baum and colleagues, they have become one of the prefered techniques in speech recognition after the implementation of Baker and Jelinek in 1970s. HMMs have been successfully applied to various other engineering problems in pattern recognition for classification and fraud detection purposes, amongst others.\\
%%TODO: ref: A tutorial on hidden Markov models and selected, Tool wear condition monitoring in drilling operations using, and Two-phase flow pattern identification using continuous hidden Markov model hidden Markov models (HMMs) and "Towards HMM based Human motion recogniion"
The type of HMM depends on the possible connections between the states. Thus, an HMM in which a state can transition to any other state is an ergodic. Other types such as the Left-Right model or Bakis do not allow all possible transitions between the states.
\subsection{HMM parameters specification}
An HMM is fully specified by the following parameters 
\begin{enumerate}
	\item N, the number of distinct states of the model. Together they form the set of individual states \(S = \{S_1, S_2, ..., S_N\}\).
	\item T, the number of observations. A sample observation sequence is denoted as \(O = \{O_1, O_2, ..., O_T\}\).
	\item \(Q = {q_t}\), the set of states with \(q_t\) denoting the current state at time instance, t such that \(q_t \epsilon S\) and \(t = 1, 2, ..., T\).
	\item K, the number of distinct observation symbols per state. 
	\item \(V = \{v_1, v_2, ..., v_K\}\), the feature set of K dimensions.
	\item \(A =  \{a_{ij} \}\), the state transition probabilities. \(a_{ij}\)  denotes probability of transitioning from state \(S_i\) to state \(S_j\).
	\item \(\Phi =   \{ \phi_{j}(k\}\), the probability distribution of observation symbols in state j.
	\item The initial state distribution, \(\pi = {\pi_i}\)
\end{enumerate}
For continuous HMM (CHMM), i.e, HMM with continuous-valued observations, \(\Phi\) consists in a probability distribution function. Many applications have succefully modelled such distributions with mixtures of Gaussian distributions. %%TODO: quote
As such, \(\phi\) is approximated by a weighted sum of M multivariate Gaussian distributions \(\eta\). For a given, observation sequence, \(\phi\)  and \(\eta\) are therefore given by equations \ref{eq:phi} and \ref{eq:eta},

\begin{align} 
	\phi(O_t) = \sum_{m=1}^M \beta_{jm} \eta(\mu_{jm}, \Sigma_{jm}, O_t), \label{eq:phi} \\
	\eta(\mu, \Sigma, O) = \frac{1}{\sqrt{(2\pi)^K|\Sigma|}}exp(-\frac{1}{2}(O-\mu)'\Sigma^{-1}(O-\mu) \label{eq:eta}
\end{align} 
\begin{align*}
	1 \leq j \leq N; 1 \leq m  \leq M; \beta_{jm} \geq 0; \sum_{m=1}^{M}\beta_{jm} = 1
\end{align*}
where \(\beta_{jm}\) is the mixture composition coefficient; \(\mu_{jm}\), \(\Sigma_{jm}\), respectively the mean vector and covariance matrix of state j; M is the number of mixture components and K is the dimensionality of O.

As a summary, the compact specification of a continous valued observation HMM is defined by \ref{eq:CHMM} and that of a discrete HMM in \ref{eq:DHMM}.
\begin{align} 
	CHMM = \lambda_C = (A, \beta_{jm}, \mu_{jm}, \Sigma_{jm}, \pi) \label{eq:CHMM} \\
	DHMM = \lambda_D = (A, b_j(k), \pi) \label{eq:DHMM}
\end{align}

\subsubsection{Basic assumptions of HMMs theory}
HMM theory is built on three basic assumptions listed below.
\begin{enumerate}
\item \textit{The Markov assumption}: HMM assumes that the probability of being in the current	at any instance of time t, is uniquely dependent on the previous state, at time, t + 1. More specifically, \(a_{ij} = P[q_t = S_j|q_{t+1}=S_i]\). This assumption makes it unsuitable for long-range correlation capturing applications.
\item \textit{The stationary assumption}: Furthermore, HMM state transition probabilities are assumed to be time-independent. Thus, the transition probabilities of two distinct time, \(t_1\) and \(t_2\) are identical, \(P[q_{t_1} = S_j|q_{t_1 -1} = S_i] = P[q{t_2}=S_i|q_{t_2-1} = S_i]\). HMMs can therefore effectively model mechanisms with stationary observations.
\item \textit{The output/observation independence assumption}: The current observation also known as emission symbol is statistically independent of the previous observations. It is "emitted" only by the current state, \(P[O|q_1, q_2, ..., q_T, \lambda]=\prod_{t=1}^{T}P[O_t|q_t, \lambda]\).
\end{enumerate} 
The three assumptions make an HMM model a relatively simple graphic modelling to be implemented. This simplicity naturally comes with some limitations in modelling more complex problems, which however, may be modelled with higher order HMMs. %%Quote
Futhermore, the three assumptions are very similar to those of a Markov chain. This is because the stochastic process of an HMM pertaining to the hidden states can be reduced to a Markov chain. In fact, an HMM is an extension of a Markov Chain. The essential difference between the two is that, with the former, there is no a one-to-one mapping between the states and the observation symbols. %%TODO: Quote biology book

\subsection{The three basics  problems for HMM design}
In %%TODO: quote: A tutorial on Hidden Markov
, Lawrence argued that an HMM design needs to answer three fondamental problems. They are the \textit{training problem}, the \textit{evaluation problem}, and the \textit{decoding problem}. Each problem and its solution is discussed in greater details next. 

\subsubsection{The evaluation problem}
The evaluation problem is about answering this question:
\textit{Given the observation sequence \(O = O_1O_2O_T\), and a model \(\lambda\), how do we efficiently compute \(P(O|\lambda\), the probability of the observation sequence?} %%TODO: quote tutorial on hmm, and others.
The naive answer to this question is simply computing the \(P(O|\lambda)\) according to equation \ref{eq:P}:
\begin{align}
	P(O|\lambda = \sum_{q_1}^{q_T}\pi_{q_1}b_{q_1}(O_1)a_{q_1q_2}b_{q_2}(O_2)...a_{q_{T-1}q_T}(O)) \label{eq:P}
\end{align}
This approach has two issues, it is not only, computationally too expensive because of the exponential complexity with respect to T, but also, intractable for very long sequence. In pactice, \(P(O|\lambda)\) is computed by an algorithm called \textit{forward-backward} procedure, which is a more efficient method.

\subsubsection{The decoding problem}
 The decoding problem can be reduced to this interrogation: \textit{Given the observation sequence \(O = O_1O_2O_T\), and the model \(\lambda\), how do we choose a corresponding state sequence \(Q = q_1q_2...Q_T\) which is optimal in some meaningful sense i.e, best "explains" the observations?}
 Simply put, this problem is about deciphering the most likely hidden states that emitted the visible observation sequences.
 This is done dynamically using the Viterbi algorithm. %%TODO: cite viterbi and explain further if need be
\subsubsection{The training problem}
Given the model, \(\lambda\), the training problem raises the following question: \textit{how do we adjust the model parameters \(\lambda\) to maximise the \(P(O|\lambda)\), the probability of the probability of the observation sequence?}
This problem is usually solved by iterative learning algorithms called expectation-maximisation. Examples of this algorithms are Baum-Welch method or any gradient based method. %%TODO: quote Tool wear condition and others
\\When using Baum-Welch algorithm, the parameters are initialised by guesses then re-estimated iteratively to find the parameters with maximum likelihood.
This method is vulnerable to local maxima issues. To avoid such cases, it is advice to run it multiple times with different initial values in order to keep the estimation with the highest likelihood value.

\subsubsection{Overfitting, order of markov, robustness: bias-var}

\section{k-Nearest Neighbour}

\section{Dimensionality reduction}
Dimensionality or dimension reduction is used pattern recognition, machine learning and statistics to find the most compact representation of the dataset by removing redundant and irrelevant information.
It is achieve by extracting principal features, i.e, feature extracting or by selecting the most relevant subset of the initial feature vector, i.e, feature selection.

\subsection{Motivations for dimensionality reduction}
When building a model, the need for dimensionality is supported by several reasons. Some of the important ones are presented in three points.
Firstly, by reducing the feature space's dimension, we can build model with higher quality. %%TODO: quote "A new hybrid filter wrapper for clustering based on ranking"
In most classification problems, the feature domains contain variables with very little to no information for the purpose at hand.
Thus, removing these features reduces the complexity of the problem which can in return, increases the model's accuracy.

Secondly, working with hundreds to thousands of features can be diffult to conceptualise and visualise. By using dimensionality reduction, we can better understand the model and present it to others by comprehensive visualisation.%%TODO: quote "A new hybrid filter wrapper for clustering based on ranking"

The third reason is about efficiency in terms of computational time and storage. In general, pattern recognition and machine learning algorithms computionally intense. Besides, storage capacity is limited in some engineering applications such as embedded systems. So, solving the problem only with the relevant features can alleviate these two problems. Consequently, the computional speed of the model can increased by using dimensionality reduction.
%%TODO: quote "Hybrid feature selection by combining filters and wrappers"
In the next section, some of the approaches to dimensionality reduction will be discussed.

\subsection{Filter methods}
\subsubsection{Forward feature subset selection}
\subsubsection{Similarity index}
\subsection{Wrapper methods}
\subsection{Feature extraction}
\subsubsection{Principal component analysis: PCA}
\subsubsection{Linear discriminant analysis: LDA}

\subsection{Hybrid filter-wrapper methods}

\section{Sufficiency of Training Data}
\section{Techniques to increase Training Data}
\subsection{Mirroring}
\include{Design}
\include{Experimental Design}
\chapter{Results} \label{sec:results}

\section{Experiment 1: Observation sequence dimensionality's effect on performance} \label{exp:feat-size}

\subsection{Aim of the experiment}
The aim of this experiment is to investigate how the number of features impacts the accuracy of a Hidden Markov Model with continuous emission symbols (CHMM), in the abscence of enough training data. Thus, the hypothesis under investigation is:\\
\textbf{\textit{In the absence of enough training data, a CHMM with observations of high dimensionality performs poorly.}}

\subsection{Experiment apparatus}
To perform this expereiment, the following materials are required:
\begin{itemize}
	\item \(\lambda\), a continuous Hidden Markov Model specified by \(\lambda = (A, \beta_{jm}, \mu_{jm}, \Sigma_{jm}, \pi)\).
	\item At least two sample data sets for training the model and testing it.
	\item A criterion to rank and select subsets of features.
	\item A measure to evaluate the performance of the CHMM model.
	\item Finally, a way to visualise the results of the experiments
\end{itemize}

\subsection{Experiment procedure}
The expreriment was performed with just the back limps by following the steps listed below:
\begin{enumerate}
	\item Step 1 - Preliminary data pre-processing: This step consisted in the data pre-processing as described in \ref{sec:pre-proc}.
	\item Step 2 - Partitioning data into training and test sets: Here, the dataset was randomly sampled into training and test sets. The training set was made relatively small, it was a sequence of 539 observations. 
	\item Step 3 -  Feature ranking: The features were ranked using separability of index \ref{sec:rank}. 
	\item Step 4 - Constructing and training the models: 18 different CHMM models, \(\lambda = {\lambda_i} = \lambda_1, \lambda_2, ..., \lambda_18\) were built and trained using the first i features of the same training dataset.
	\item Step 5 - Testing and evaluation the models: After training the models, they were tested with the same test dataset with the correct feature subsets and their performance were evaluated in terms of decoding accuracy and log-likelihood value. 
	\item The different accuracies and the sequence log-likelihood values were finally plotted as a function of the observation dimensionality. Moreover, the observations were grouped based on the corresponding hidden state sequence and scattered in a 2-dimensional principal component space. This is to compare the decoded states against the ground-truth.
\end{enumerate}

\subsection{Experiment results}
The results of the experiments are presented in figure \ref{fig:dim-acc}, \ref{fig:dim-log}, \ref{fig:gt-5dim}, \ref{fig:es-5dim}, \ref{fig:gt-18dim}, \ref{fig:es-18dim}.\\
Figure \ref{fig:dim-acc} and \ref{fig:dim-log} respectively show how the hidden state decoding accuracy and the test sequence log-likelihood estimated by the CHMM model varies as the the number of features considered increases.\\
Figure \ref{fig:gt-5dim} and \ref{fig:es-5dim} are visualisations of the 5-dimensional observation sequence grouped respectively, according to the actual state sequence and the decoded state sequence. Figure \ref{fig:gt-18dim} and \ref{fig:es-18dim}, are for an 18-dimensional observation.
5 and 18 dimensions were presented because they resulted in the two extreme prediction accuracies. The results for other dimensions may be found in the appendix from figure \ref{fig:gt-1dim} to figure \ref{fig:es-17dim}, %%TODO: ref appendices
\begin{figure}[ht!]
	\centering
	\includegraphics[width=\textwidth,height=\textheight,keepaspectratio]{Figures/dimensionality-effect-acc}
	\caption{The effect of CHMM's observation dimensionality the state sequence decoding accuracy}
	\label{fig:dim-acc}
\end{figure}

\begin{figure}[ht!]
	\includegraphics{Figures/dimensionality-effect-log}
	\caption{The effect of CHMM's observation dimensionality the log-likelihood}
	\label{fig:dim-log}
\end{figure}


\begin{figure}[ht!]
	\includegraphics{Figures/ground-truth-scatter-with-5-features}
	\caption{Scatter plot of  5-dimensional observations grouped according the ground-truth state sequence}
	\label{fig:gt-5dim}
\end{figure}
\begin{figure}[ht!]
	\includegraphics{Figures/estimation-scatter-with-5-features}
	\caption{Scatter plot of 5-dimensional observations grouped according to the estimated state sequence}
	\label{fig:es-5dim}
\end{figure}

\begin{figure}[ht!]
	\includegraphics{Figures/ground-truth-scatter-with-18-features}
	\caption{Scatter plot of 18-dimensional observations grouped according to the ground-truth state sequence}
	\label{fig:gt-18dim}
\end{figure}

\begin{figure}[ht!]
	\includegraphics{Figures/estimation-scatter-with-18-features}
	\caption{Scatter plot of 18-dimensional observations grouped according to the estimated state sequence}
	\label{fig:es-18dim}
\end{figure}

\subsection{Analysis and discussion of results}

Starting with 1 feature, the CHMM model's state decoding accuracy in \ref{fig:dim-acc} increases. It reaches its maximum performance at about 73\% accuracy with 5 features. After 5 features, the accuracy generally declines to about just 40\% accuracy with all 18 features. The decline is however not consistent, for example, there is a local peak at 13 and 17 features.
A performance increase of up to 78\% was obtained by reducing the feature set's dimension by from 18 to 5.
In addition, the sequence log-likelihood value is close to zero from 1 to 8 features. It then slowly declining - in overall - from 9 to 15 features and suddenly gets to its lowest value at 18 dimensions.
The log-likelihood being a measure of the probability that the test sequence was generated by the model, it is clear that the CHMM model better recognises test sequence with smaller feature sizes.\\
Thus, the smaller the feature size, the better the CHMM performance with a relatively small training dataset. The model is both better at decoding the hidden states as well as recognising IMU measurements generated by the dog's gait mechanism.
This fact is further shown by \ref{fig:gt-5dim}, \ref{fig:es-5dim}, \ref{fig:gt-18dim} and \ref{fig:es-18dim}. With 5 features, the model correctly attributes most IMU measurements to the correct footfalls except the state 4 observations which were mostly classified as state 3 emissions. In opposite, with 18 features, the HMM model predicted wrongly that the dog remained in state 3 during all the 539 sequences. In fact, with 18 dimensions, the model is not twice as good as predicting with random guesses, i.e, \(0.4 < 2\times\frac{1}{4}\).

\subsection{Conclusions and recommendations of the experiment}
Based on the above results and discussions, it can be concluded that in the absence of enough training data, a CHMM performs very poorly. This validates the hypothesis formulated. It is therefore necessary to further explore the effect of dimensionality reduction with more robust techniques. This investigation is performed in the next experiment with various dimensionality reduction technique and varying data sizes.


\section{Experiment 2: The impact of dimensionality reduction on performance}  \label{exp:dim}

\subsection{Aim of the experiment}
The aim of this experiment is to investigate the effect of dimensionality reduction on the performance of a continuous Hidden Markov Model (CHMM).
The hypothesis under investigation is therefore the following:
\textbf{\textit{Dimension reduction can cause an increase in a CHMM's performance when there is not enough training data.}}
Thus, the performances of the CHMM with and without dimensionality reduction are compared to test the hypothesis.

\subsection{Experiment apparatus}
The assets needed to perform the experiment are listed below.
\begin{itemize}
	\item \(\lambda\), a continuous Hidden Markov Model specified by \(\lambda = (A, \beta_{jm}, \mu_{jm}, \Sigma_{jm}, \pi)\).
	\item Four distinct dimensionality reduction methods. Two feature extraction methods and two feature selection methods were considered. The feature extraction methods were Principle Component Analysis (PCA), Linear Discriminant Analysis (LDA). The two feature selection methods were feature ranking with separability index denoted SI Ranking, and a combination of forward feature selection and separability index denoted SI-forward.
	\item A performance measure to compare the different models with and without dimensionality reduction. Again, the metrics used were the hidden state decoding accuracy and the ability to recognise a sequence generated by the model namely the log-likelihood.
\end{itemize}

\subsection{Experiment procedure}
The experiment was performed as follows.
Firstly, the dataset was partitioned into two different set for training and testing using random sampling.\\
Using the same training dataset, five specific models were built and trained. They are denoted as: \(\lambda_{No Reduction}, \lambda_{PCA}, \lambda_{LDA}, \lambda_{SI}, \lambda_{SI-forward}\), respectively, for the model with all 18 features, the models built with PCA, LDA, SI Ranking and SI-forward. Regarding the last four models, the dimension of the training set was reduced with the corresponding technique before feeding it into the actual CHMM model. (Please refer to \ref{sec:dim-red} for more details on the dimension reduction methods design)\\
Next, the different models were tested with the same test dataset - naturally applying the dimensionality reduction technique where required- and the prediction accuracy and the log-likelihood values were recorded.\\
This above procedure was repeated while varying the proportion of training data used from \(10\%\) to \(90\%\) of the total dataset.\\
The prediction accuracies and log-likelihoods were finally plotted as a function of the training data size for each model.
These findings are presented in the figures \ref{fig:size-acc} and \ref{fig:size-log}. \\\\
In order to verify, the variance in the five distinct models, a separate an additional experiment was performed for a bias-variance error analysis. Thus, five models were trained and tested 100 times with different datasets. For each model, the average accuracy over the 100 tests was taken as its bias error and variance in the 100, as the variance error.

\subsection{Experiment results}
Firstly, figure \ref{fig:size-acc} illustrates how the performances of the five CHMMs compare against each other as the training data size increases.
Secondly, the log-likelihoods presented in \ref{fig:size-log} shows how effectively each model can recognise an observation sequence generated by the underlying mechanism. These results are discussed in the next sub-section.
Finally, \ref{fig:bias-var} is the bias-variance errors of the different models.

\begin{figure}[ht!]
	\includegraphics{Figures/size-acc-2}
	\caption{The effect of training datasize on the prediction accuracy}
	\label{fig:size-acc}
\end{figure}

\begin{figure}[ht!]
	\includegraphics{Figures/size-log-2}
	\caption{The effect of training datasize on the log likelihood}
	\label{fig:size-log}
\end{figure}

\begin{figure}[ht!]
	\centering
	\includegraphics{Figures/bias-var}
	\caption{Bias-Variance tradeoff analysis}
	\label{fig:bias-var}
\end{figure}

\subsection{Analysis and discusion of results}
In the sub-sections below, the effect of each dimensionality reduction technique is discussed.

\subsubsection{No reduction vs feature Ranking: SI Ranking}
The two graphs under considering here are the purple and the blue graphs in \ref{fig:dim-acc} and \ref{fig:dim-log}. \\
For a relatively small training data size, i.e, under 1000 samples, the model with feature subset selection using feature ranking outperforms the model with dimensionality reduction. This fact is true for both the prediction accuracy and log-likelihood values. At 539 sample data, there is at least 8\% improvement after reducing the feature size in classification. Based on the linear nature of the graphs before 1000, it could be extrapolated that this improvement will be more significant for observation sequences with less than 539 samples.
After, 1000 samples, the accuracies of both models increases but, the model built with the 18-dimensional observation sequence increasing performs better.
%It might, therefore, not be necessary to perform any dimensionality reduction if improving the accuracy is the only motivation when the training data size is greater than 1000 instances. This is more so because, variance error as shown in \ref{fig:bias-var} are close to zero for both models.

\subsubsection{No reduction vs forward feature selection: SI-forward}
Here, we are discussing the green and the blue graphs in \ref{fig:dim-acc} and \ref{fig:dim-log}. From these two graphs, we can note that the forward feature selection has the similar effect to the feature ranking. This is because they are both feature subset selection methods. It is therfore a confirmation of the above discussion.\\
From these similar results, we can conclude that the very fast feature ranking \ref{sec:rank} method can be used in lieu of the slower forward feature selection method \ref{sec:forw}. This result confirms the efficacity of the 'feature classifiability' \ref{con:class} as feature ranking criterion for classification applications.

\subsubsection{No reduction vs PCA and LDA}

The two feature extraction methods decreased the performance of the models for all data sizes. In addition, their performances do not consistently increase with the increase in data size. The nature of IMU measurements is not suited to these two feature extraction methods. The experiment could have been better with other successful feature extraction methods found in literature such as FFT \cite{towa2009},
Time-frequency domain analysis \cite{ches2012}.

\subsubsection{Bias - variance error analysis}
The presented results in \ref{fig:bias-var} is for a dataset with 539 samples.
The bias error for all five models is relatively low. This because the test data was similar to the training data. 
Interestingly, the variance in the training errors over the 100 differet runs are all very close to zero. This is not very surprising because the model's parameters, particularly the mean and the co-variance matrices were designed to have low variance. %%TODO: have a section on robustness of the mean and variance in the evaluation or testing section.
By looking at the test errors and the bias errors, the best model appears to be the model built with the forward feature selection method, \(\lambda_{SI-forward}\) when the dataset size is small. The next comparable model would be the model built with the feature ranking metho, i.e, \(\lambda_{SI}\). So, with a very strict constraint on the model's speed, \(\lambda_{SI}\) should be favoured because, it is significantly faster than \(\lambda_{SI-forward}\).

\subsection{Conclusions and recommendations of the experiment}
Considering the above findings, we can effectively confirm that dimensionality reduction can effectively improve the performance of an IMU based CHMM when the dataset is not large enough. However, this does not apply to every dimensionality reduction method. A large range of technique should be explored to select the most appropriate such as the forward feature selection and the feature ranking methods.


\section{Experiment 3: The necessity of combining the front and back IMU sensor measurements}

\subsection{Aim of the experiment}
The objective of this experiment is to determine whether or not it is necessary to consider both the front and back IMU measurements when predicting only the front or back footfalls. Thus, the following hypothesis was constructed:\\
\textbf{\textit{When predicting only the front or back footfalls of the dog, building the model with the aggregated front and back IMU measurements can improve its performance.}}

\subsection{Experiment apparatus}
To perform this experiment, the following materials are required:
\begin{itemize}
	\item \(\lambda_f\) and \(\lambda_b\), two continuous Hidden Markov Models for the front and back limbs respectively specified by \(\lambda_f = (A_f, \beta_{jm_f}, \mu_{jm_f}, \Sigma_{jm_f}, \pi_f)\) and \(\lambda_b = (A_b, \beta_{jm_b}, \mu_{jm_b}, \Sigma_{jm_b}, \pi_b)\). These models will be further split into two.
	\item Four data samples: two distinct training sets and two distinct test sets for the two models, where each contains both the front and back IMU measurements.
	\item A measure to evaluate the performance of the CHMM model.
	\item Finally, a way to visualise the results of the experiments
\end{itemize}

\subsection{Experiment procedure}
To verify our hypothesis, similar experiments to predict the back and front footfalls of the dogs were performed. Further details are given next.\\
First of all, two front and back models: \(\lambda_{f_{combined}}\) and \(\lambda_{b_{combined}}\) were constructed by using the combined IMU data from two sensor sets.\\
Furthermore, two other front and back models: \(\lambda_{f_{just-front}}\) and \(\lambda_{b_{just-back}}\) were built using only the front IMU and back IMU measurements, respectively. \\Thus, we have two couples of models to be compared: \(( \lambda_{f_{combined}} vs. \lambda_{f_{just-front}})\) and \((\lambda_{b_{combined}} vs. \lambda_{b_{just-back}})\) \\
Secondly, the models to be compared were trained using the same training dataset, where the back or front IMU measurements were removed in the respective case of \(\lambda_{f_{just-front}}\) and \(\lambda_{b_{just-back}}\). \\
Finally, the models were tested with their corresponding training datasets. This was done purposefully since the relative comparison of the models with 'combined' and 'separate' IMU data is the subject matter, not the individual performances.\\
The experiment was repeatedly performed while varying the size of the training dataset. The state decoding accuracies and the corresponding log-likelihood were recorded. They are presented in the following sub-section.

\subsection{Experiment results}
The results of the current experiment are presented in \ref{fig:front-comb-acc}, \ref{fig:front-comb-log}, \ref{fig:back-comb-acc} and \ref{fig:back-comb-log}. In the same order, they represent the front footfalls decoding accuracy, the front sequence log-likelihood,  back footfalls decoding accuracy and the back sequence log-likelihood.

\begin{figure}[ht!]
	\includegraphics[width=\textwidth,height=\textheight,keepaspectratio]{Figures/front-comb-acc}
	\caption{Front footfalls prediction accuracy of both IMUs vs with only the front IMU}
	\label{fig:front-comb-acc}
\end{figure}

\begin{figure}[ht!]
	\includegraphics[width=\textwidth,height=\textheight,keepaspectratio]{Figures/front-comb-log}
	\caption{Front footfalls prediction log-likelihood with both IMUs vs with only the front IMU}
	\label{fig:front-comb-log}
\end{figure}

\begin{figure}[ht!]
	\includegraphics[width=\textwidth,height=\textheight,keepaspectratio]{Figures/back-comb-acc}
	\caption{Back footfalls prediction accuracy with both IMUs vs with only the front IMU}
	\label{fig:back-comb-acc}
\end{figure}

\begin{figure}[ht!]
	\includegraphics[width=\textwidth,height=\textheight,keepaspectratio]{Figures/back-comb-log}
	\caption{Back footfalls prediction log-likelihood with both IMUs vs with only the front IMU}
	\label{fig:back-comb-log}
\end{figure}


\subsection{Analysis and discussion of results}

Let's proceed by analysing the front and back footfalls separately before putting them together with conclusive thoughts.

\subsubsection{Front footfalls}
The experiment revealed that for both the combined and just the front IMU data, the classification accuracy increases with the increase in the training data size as testified by \ref{fig:front-comb-acc}. Overall, building the model with just front sensor measurements achieves better accuracy except with about 3000 and 4500 observation samples.\\
The difference in performance is more significant with smaller data sizes. For instance, with 539 samples, the combination resulted in a very poor performance, below 50\% accuracy, whereas, taking just the front measurements resulted in about 63\%. This is at least 35\% improvement.
This difference is due to the dimensionality of the observation sequences as demonstrated in \ref{exp:feat-size}. With just the front measurements, we are dealing with 9 features against 18 features when the front and back measurements are considered.
After about 3000 samples, the two models prediction accuracies become comparable. This is because there is sufficiently enough training data to cater for the high dimensionality.\\\\

The ability of the two models to recognise the IMU measurements generated by the dog are very comparable from 1000 samples onwards as shown in \ref{fig:front-comb-log}. Before 1000 samples, the model with just the front sensor measurements gives a higher likelihood value, constantly close to zero. This confirms the trend in the prediction accuracy.

\subsubsection{Back footfalls}

The back footfalls prediction with just the back sensor data and both the front and back sensor data show similar results to the front footfall case. However, the model with the combined sensors measurements catches up quicker at around 1000 samples. It even shows slightly better performance in accuracy after 2500 samples.\\\\
The common observation about both the front and back footfalls is the following. Only, the front and the back sensor measurements can better predict or recognise the front and back footfalls respectively, when the available observation sequence is relatively small, below 1000.
This finding is advantageous for several reasons. In fact, this realisation can be used to reduce the dimension of the data from 18 down to 9 when dealing with a small dataset. On one hand, this is not only good in terms of the model's accuracy. On the other hand, when dealing with just the front or the back legs, the required logistics such as the memory size, the unnecessary back or front sensors themselves. 

\subsection{Conclusions and recommendations of the experiment}
In light of the finding above, it can be concluded that: using all the front and back IMU measurements to predict just the front and back footfalls does not necessarily improves the performance. It degrades the performance when there is not enough training data because of the high dimensionality.
%The formulated hypothesis is, therefore, rejected based on this experiment.\\
An improvement to this experiment would be to explore how the dimensionality reduction of the different datasets would influence the outcome. Moreover, a futher investigation how well can the back sensors measurement be used to predict the front footfalls. Similarly, how well can the front sensors be used to predict the back footfalls.


\section{Experiment 4: Motion type recognition}  \label{exp:motion}

\subsection{Aim of the experiment}
This experiment is about using IMU based HMM to identify a particular action of the dog. The three actions considered are: running, walking, and trotting.
The hypothesis under investigation is, therefore, the following:
\textbf{\textit{IMU based HMMs can be used to successfully perform action recognition.}}

\subsection{Experiment apparatus}
The main tools required to perform the experiment are listed next. Note that the experiment was performed for the front limbs only.
\begin{itemize}
	\item Three distinct CHMM models: \(\lambda_{running}\), \(\lambda_{walking}\) and \(\lambda_{trotting}\) to model the dog's run, walk, and trot.
	\item Three separate IMU datasets for the three action denoted by \(D_r, D_w, D_t\) respectively for running, walking and trotting.
	\item An action recognition criterion. In this experience, the prediction accuracies confusion matrix and the log-likelihood values were used.
\end{itemize}

\subsection{Experiment procedure}
The following steps were followed to run the experiment.

\begin{enumerate}
	\item Extracting the three distinct datasets: This step is very similar to the pre-processing described here \ref{sec:pre-proc}. The only difference is, the data for the dog's run, walk and trot were separated to obtain the three distinct sets. The initial sequences were mirrored three times for running walking motions and four times for the trotting action, to obtained large enough samples (please refer to this section \ref{sec:mirror} for more details on the mirroring process).
	
	\item Constructing and training the models: 65\% of the three distinct datasets were used to build and estimate the parameters of \(\lambda_{running}\), \(\lambda_{walking}\) and \(\lambda_{trotting}\).
	
	\item Testing and evaluating the models: Each model was tested three times with the remaining 35\% of the three distinct datasets. The prediction accuracies and the log-likelihoods were recorded and tabulated as presented in the results sub-section.\\
	The log-likelihood value \ref{tab:action-recogn-log} measure the likelihood of the IMU measurement sequence having been emitted by the CHMM model in question.
	Thus, for a given sequence of measurements, can be attributed to the model that outputs the highest log-likelihood value \cite{cont2013}.
	The same purpose can be achieved by using the prediction accuracy confusion matrix, where the best candidate model is the model  with the highest accuracy for given an observation sequence.
\end{enumerate} 

\subsection{Experiment results}

Table \ref{tab:action-recogn-log}: the hidden state decoding accuracy and \ref{tab:action-recogn-acc}: the log-likelihood values, summary the results of the experiment.\\

\begin{table}[h!] 
	\centering
	\begin{tabular}{ |c|c|c|c|} 	
		\hline	
		& \(D_r\) &  \(D_w\) & \(D_t\)\\ 
		\hline
		\(\lambda_{running}\) & 91.16\% & 2.06\% & 0.22\% \\ 
		\hline
		\(\lambda_{walking}\) & 21.06\% & 100.00\% & 75.53\% \\ 
		\hline
		\(\lambda_{trotting}\) & 27.40\% & 45.72\% & 100.00\%\\
		\hline	   	
	\end{tabular}
	\caption{Footfall prediction confusion matrix (\% accuracy )}
	\label{tab:action-recogn-acc}
\end{table}

\begin{table}[h!] 
	\centering
	\begin{tabular}{ |c|c|c|c|} 	
		\hline	
		& \(D_r\) &  \(D_w\) & \(D_t\)\\ 
		\hline
		\(\lambda_{running}\) & 0.00  & -0.00\(\times10^{14}\)   & -0.00\(\times10^{14}\)\\ 
		\hline
		\(\lambda_{walking}\)  & -0.00\(\times10^{14}\)  &  0.00  & -0.00\(\times10^{14}\)\\ 
		\hline
		\(\lambda_{trotting}\)  & -1.44\(\times10^{12}\)  &  -0.1302  & 0.00\\
		\hline	   	
	\end{tabular}
	\caption{Footfall sequence log-likelihood}
	\label{tab:action-recogn-log}
\end{table}

\subsection{Analysis of results}
By inspection, it can be observed that, the principal diagonal of the confusion matrix generates the highest accuracies in \ref{tab:action-recogn-acc}. It can, therefore, be concluded that the different models effectively recognise their corresponding motion type. This fact is confirmed by the log-likelihood table with the highest values on the main diagonal. Please note that the values -0.00\(\times10^{14}\) are very tiny non-zero quantities.\\
It is worth mentioning the 100\% accuracies when the unseen walking and trotting dataset were used to test their corresponding models. Strictly speaking, the 100\% accuracies might have been obtained because the transition errors were ignored (ref. \ref{sec:eval}). \\ Regardless, this is a better performance on unseen dataset compared to the performance seen thus far when the dataset from three actions are aggregated (ref. \ref{sec:aggregate}). Naturally, the non-aggregated training dataset better models a particular action than building a common model that fits all the three action types.

\subsection{Conclusions of the experiment}
On the account of the findings discussed in the above subsection, the following conclusions can be drawn.
Indeed, IMU based CHMM can be used to successfully perform motion type recognition. 


%\newpage
%\section{Experiment 5: Comparing the back and front gait of the dog}  \label{exp:front-back}
%
%\subsection{Aim of the experiment}
%The present experiment is very similar to motion recognition experiment in \ref{exp:motion} in nature but, it differs in purpose. Here, the aim is to study the similarities and differences between the front and back limb movements of a dog, given a specific action. The dog's run, walk and trot where investigated.
%The hypothesis below was therefore formulated.
%\textbf{\textit{The front and back limbs of a dog exhibit gait patterns.}}
%
%\subsection{Experiment apparatus}
%The assets needed to perform the experiment are listed below.
%\begin{itemize}
%	\item \(\lambda\), a continuous Hidden Markov Model specified by \(\lambda = (A, \beta_{jm}, \mu_{jm}, \Sigma_{jm}, \pi)\).
%
%\end{itemize}
%
%\subsection{Experiment procedure}
%
%\subsection{Experiment results}
%
%\begin{table}[h!] 
%	\centering
%	\begin{tabular}{ |c|c|c|} 	
%		\hline	
%		\textbf{Body part} & \textbf{Front footfalls (\%)} &  \textbf{Back footfalls (\%)}\\ 
%		\hline
%		Front footfalls & 88.9210 & 17.4678\\ 
%		\hline
%		Back footfalls & 14.3545 & 89.5708 \\ 
%		\hline	   	
%	\end{tabular}
%	\caption{Prediction accuracies for running}
%	\label{tab:front-back-run-acc}
%\end{table}
%
%\begin{table}[h!] 
%	\centering
%	\begin{tabular}{ |c|c|c|} 	
%		\hline	
%		\textbf{Body part} & \textbf{Front footfalls} &  \textbf{Back footfalls}\\ 
%		\hline
%		Front footfalls &  8.4012x\(10^{-4}\) & 4.2653x\(10^{-4}\)\\ 
%		\hline
%		Back footfalls & 7.9955x\(10^{-4}\) & 8.6471x\(10^{-4}\) \\ 
%		\hline	   	
%	\end{tabular}
%	\caption{Log-likelihood for running}
%	\label{tab:front-back-run-log}
%\end{table}
%
%
%\begin{table}[h!] 
%	\centering
%	\begin{tabular}{ |c|c|c|} 	
%		\hline	
%		\textbf{Body part} & \textbf{Front footfalls} &  \textbf{Back footfalls}\\ 
%		\hline
%		Front footfalls &  100.0000x\(10^{-4}\) & 55.8629x\(10^{-4}\)\\ 
%		\hline
%		Back footfalls & 55.7803x\(10^{-4}\) & 100.0000x\(10^{-4}\) \\ 
%		\hline	   	
%	\end{tabular}
%	\caption{Prediction accuracies for walking}
%	\label{tab:front-back-walk-acc}
%\end{table}
%
%\begin{table}[h!] 
%	\centering
%	\begin{tabular}{ |c|c|c|} 	
%		\hline	
%		\textbf{Body part} & \textbf{Front footfalls(\%)} &  \textbf{Back footfalls(\%)}\\ 
%		\hline
%		Front footfalls & 2.6346x\(10^{-4}\) & 2.6346x\(10^{-4}\)\\ 
%		\hline
%		Back footfalls & 2.6025x\(10^{-4}\) & 2.6025x\(10^{-4}\)\\ 
%		\hline	   	
%	\end{tabular}
%	\caption{Log-likelihood for walking}
%	\label{tab:front-back-walk-log}
%\end{table}
%
%
%\begin{table}[h!] 
%	\centering
%	\begin{tabular}{ |c|c|c|} 	
%		\hline	
%		\textbf{Body part} & \textbf{Front footfalls (\%)} &  \textbf{Back footfalls(\%)}\\ 
%		\hline
%		Front footfalls &  100.0000 & 74.6082\\ 
%		\hline
%		Back footfalls & 74.6082 & 100.0000 \\ 
%		\hline	   	
%	\end{tabular}
%	\caption{Prediction accuracies for trotting}
%	\label{tab:front-back-trot-acc}
%\end{table}
%
%
%\begin{table}[h!] 
%	\centering
%	\begin{tabular}{ |c|c|c|} 	
%		\hline	
%		\textbf{Body part} & \textbf{Front footfalls} &  \textbf{Back footfalls}\\ 
%		\hline
%		Front footfalls & 7.0234x\(10^{-4}\) & 7.0234x\(10^{-4}\)\\ 
%		\hline
%		Back footfalls & 6.8193x\(10^{-4}\) & 6.8193x\(10^{-4}\)\\ 
%		\hline	   	
%	\end{tabular}
%	\caption{Log-likelihood for trotting}
%	\label{tab:front-back-trot-log}
%\end{table}
%
%\subsection{Analysis of results}
%
%\begin{table}[h!] 
%	\centering
%	\begin{tabular}{ |c|c|c|} 	
%		\hline	
%		\textbf{Body part} & \textbf{Front footfalls (\%)} &  \textbf{Back footfalls (\%)}\\ 
%		\hline
%		Front footfalls & 1 & -1\\ 
%		\hline
%		Back footfalls & -1 & 1 \\ 
%		\hline	   	
%	\end{tabular}
%	\caption{Prediction accuracies correlation for running}
%	\label{tab:front-back-run-acc-corr}
%\end{table}
%
%\begin{table}[h!] 
%	\centering
%	\begin{tabular}{ |c|c|c|} 	
%		\hline	
%		\textbf{Body part} & \textbf{Front footfalls} &  \textbf{Back footfalls}\\ 
%		\hline
%		Front footfalls & 1 & -1\\ 
%		\hline
%		Back footfalls & -1 & 1 \\ 
%		\hline	   	
%	\end{tabular}
%	\caption{Prediction log-likelihood correlation for running}
%	\label{tab:front-back-run-log-corr}
%\end{table}
%
%
%\begin{table}[h!] 
%	\centering
%	\begin{tabular}{ |c|c|c|} 	
%		\hline	
%		\textbf{Body part} & \textbf{Front footfalls} &  \textbf{Back footfalls}\\ 
%		\hline
%		Front footfalls &  1 & -1\\ 
%		\hline
%		Back footfalls & -1 & 1 \\ 
%		\hline	   	
%	\end{tabular}
%	\caption{Prediction accuracies correction for walking}
%	\label{tab:front-back-walk-acc-corr}
%\end{table}
%
%\begin{table}[h!] 
%	\centering
%	\begin{tabular}{ |c|c|c|} 	
%		\hline	
%		\textbf{Body part} & \textbf{Front footfalls(\%)} &  \textbf{Back footfalls(\%)}\\ 
%		\hline
%		Front footfalls & 1 & 1\\ 
%		\hline
%		Back footfalls & 1 & 1\\ 
%		\hline	   	
%	\end{tabular}
%	\caption{Prediction log-likelihood correlation for walking}
%	\label{tab:front-back-walk-log-corr}
%\end{table}
%
%
%\begin{table}[h!] 
%	\centering
%	\begin{tabular}{ |c|c|c|} 	
%		\hline	
%		\textbf{Body part} & \textbf{Front footfalls (\%)} &  \textbf{Back footfalls(\%)}\\ 
%		\hline
%		Front footfalls &  -1 & 1\\ 
%		\hline
%		Back footfalls & 1 & -1 \\ 
%		\hline	   	
%	\end{tabular}
%	\caption{Prediction accuracies correlation for trotting}
%	\label{tab:front-back-trot-acc-corr}
%\end{table}
%
%
%\begin{table}[h!] 
%	\centering
%	\begin{tabular}{ |c|c|c|} 	
%		\hline	
%		\textbf{Body part} & \textbf{Front footfalls} &  \textbf{Back footfalls}\\ 
%		\hline
%		Front footfalls & 1 & 1\\ 
%		\hline
%		Back footfalls & 1 & 1\\ 
%		\hline	   	
%	\end{tabular}
%	\caption{Prediction log-likelihood correlation for trotting}
%	\label{tab:front-back-trot-log-corr}
%\end{table}
%
%\subsection{Conclusions and recommendations of the experiment}
\chapter{General discussions}

Different experiments were performed and presented in the results section \ref{sec:results}. This section further discusses the findings under the following headings.

\section{Data dimensionality and bias analysis}

This secetion is based on experiment 1 \ref{exp:feat-size} and 2 \ref{exp:dim}.
From the two experiments, it is clear that with a limited-size training data, building the model will all 18 features results in a poor performance. Thus, dimensionality reduction needs to be used to increase the model's precision. The best technique to do this is the combination of forward feature selection and feature ranking using separability index. More than 80\% classification accuracy was obatained with four selected features using this filter technique.\\
However, up to 95\% accuracy was obtained without dimensionality reduction when the training data size is relatively large. Therefore, it may not be necessary to reduce the feature size with enough data set, around 2000 samples. The robustness of the model without dimensionality reduction backs this statement. In fact, the variance error was extremely low both models: with and without dimensionality reduction. This is not surprising for two different reasons. Inherently, HMMs are very robust because they are probabilistic models. Moreover, the estimated means and the co-variances are the two potential sources of bias in the model. But, the mean is an unbiased estimator as proven in \ref{eq:unbiased-mean} 
\begin{align} 
	E[\mu^*] = E[\frac{1}{N}\sum_{k=1}^{N}X^k] = \frac{1}{N}\sum_{k=1}^{N}E[X^k]=\mu \label{eq:unbiased-mean}
\end{align} 
Furthermore, although variance is a biased estimator, they were normilised with N-1. Thus, with a large N value, the bias tends to zero as shown here in \ref{eq:biased-var} and \ref{eq:unbiased-var}. So, with a large dataset, the gait sequence estimator is unbiased when the observation sequence is long and biased, otherwise. In fact, with small training data sequence, the bias increases with the feature size. This explains the very poor performance of the model is built with 539 18-dimenstional observations.
 \begin{align} 
	Biased \quad	E[\sigma^{2^*}] = E[\frac{1}{N}\sum_{k=1}^{N}(X^k - \mu*)^2] = \frac{N - 1}{N}\sigma^2 \label{eq:biased-var}\\
 \quad with \quad a \quad large \quad N, \quad E[\sigma^{2^*}] = \sigma^2 \label{eq:unbiased-var} \quad becomes \quad unbiased
 \end{align}

\section{Data aggregation and mirroring}
Overall, the different models performed better with the increase in training data size. This fact shows that the data aggregation and the mirroring techniques worked very well. Nevertheless, the aggregation of the data from all the action types led to a loss of specificity as expected, therefore, decreasing the model's precision. This fact was uncovered by the motion recognition experiment \ref{exp:motion}. In this experiment, when specific models were built for each action, the prediction accuracies in table \ref{tab:action-recogn-acc} were very close to 100\%.\\
Even though the reverse sequences were appended, the models still performed very well when tested with the correct data. This testifies that the increase of data by mirroring does not reduce performance, contrary to the aggregation method.
\section{State duration time}
\chapter{Conclusions}
The present report consolidates the work done on gait sequence modelling and estimation using Hidden Markov Models (HMMs). 
We went from the continuous IMU measurements of a dog moving, to gait state identification using HMM with Gaussian mixture density functions.\\
Given limitation imposed by the small size of the dataset, dimensionality reduction techniques were explored. In addition, data aggregation and mirroring techniques were employed to increase the data. After successfully implementing the various models, experiments were designed and performed to test the hypotheses formulated in the introduction. These hypotheses are repeated here as a reminder.
\begin{enumerate}
	\item HMMs can successfully model gait sequence dynamics using IMU data, in the absence of huge training data.
	\item Data aggregation and mirroring techniques can be used to overcome training data size limitations.
	\item Dimensionality reduction can increase the performance of an HMM, in the absence of a large training set.
	\item HMMs can be used to successfully perform gait activity recognition.
\end{enumerate}
In light of the above results and the discussions, these hypotheses can be confidently accepted. \\
Indeed, continuous HMMs with Gaussian mixture can be used to build robust gait sequence estimators from IMI measurements with up to 95\% precision.
This accuracy is obtainable by aggregating the measurements of the different gait actions and/or by using considering the reverse gait sequence as a valid sequence.\\
Furthermore, when dealing with a small training data, dimensionality reduction can increase the model's performance by up to 78\%. However, with a large dataset, it might not be necessary to perform any reduction with just 18-dimensions. This last conclusion should be taken with some degree of reservation.
Finally, the last experiment showed that IMU based HMM can be used for gait type classification.\\
The objectives of this project can, therefore, be considered met. In the next section recommendations are made for possible improvements and future works.
\chapter{Recommendations for future work}
In this chapter are presented the recommendations that follow from the completion of the project. These recommendations can serve as an extension of the project for improvement and future work.
\begin{itemize}
	\item \textit{Online real-time testing:} Even though the model was tested with unseen data, an online real-time implementation and testing on a moving dog would testify of the reliability of the model built.
	
	\item \textit{Gather more data from different dogs:} The HMM model was formulated and the experiments were performed based on measurements from a single subject. More data should be gathered from multiple dogs to test the present and subsequently tune the model's parameters based on the outcome of the tests.
	
	\item \textit{Incorporate more domain knowledge:} Given the limited amount of time, domain-specific knowledge was not thoroughly explored. Making more use of available knowledge on quadruped movement may improve the model's complexity and performance. 
	
	\item \textit{Compare the HMM models to other algorithms:} The performance of the constructed models should be compared to other classification and pattern recognition methods.
	
	\item \textit{Combine the front and back HMMs:} In this design, separate models were built for the front and the back legs of the dog. Although this made the algorithm applicable to bipedal gait estimation, it would be useful to combined the two models and evaluate the performance of the holistic 16-states HMM.
	
	\item \textit{Investigate more dimensionality reduction techniques:} The feature extraction methods performed did not perform very well, more dimensionality reduction methods such as Fast Fourier Transform (FFT) \cite{towa2009}, time-frequency analysis \cite{ches2012} should be investigated.
\end{itemize}



\begin{thebibliography}{5}
	
\bibitem{tuto1989} L. R. Rabiner, "A tutorial on hidden Markov models and selected applications in speech recognition", \emph{Proceedings of the IEEE}, vol. 77, pp. 257-286, 1989, ISSN 0018-9456.
\bibitem{cont2013} Ghazaleh Panahandeh, Nasser Mohammadiha, Arne Leijon, Peter Handel, "Continuous Hidden Markov Model for Pedestrian Activity Classification and Gait Analysis", \emph{Instrumentation and Measurement IEEE Transactions on}, vol. 62, pp. 1073-1083, 2013, ISSN 0890-6955, DOI 10.1109/5.18626
\bibitem{towa2009} Guangyi Shi and Yuexian Zou and Yufeng Jin and Xiaole Cui and W. J. Li, "Towards HMM based human motion recognition using MEMS inertial sensors", \emph{2008 IEEE International Conference on Robotics and Biomimetics}, pp. 1762-1766, 2009, DOI 10.1109/ROBIO.2009.4913268.
\bibitem{ches2012} G. Panahandeh and N. Mohammadiha and A. Leijon and P. Handel, "Chest-mounted inertial measurement unit for pedestrian motion classification using continuous hidden Markov model", \emph{2012 IEEE International Instrumentation and Measurement Technology Conference Proceedings}, pp. 991-995, 2012, ISSN 1091-5281, DOI 10.1109/I2MTC.2012.6229380.
\bibitem{thre_2001} I. P. I. Pappas, M. R. Popovic, T. Keller, V. Dietz, and M. Morari, “A
reliable gait phase detection system,” \emph{IEEE Trans. Neural Syst. Rehabil.
	Eng.}, vol. 9, no. 2, pp. 113–125, Jun. 2001.
\bibitem{fuzz_} C. Senanayake and S. Senanayake, “Computational intelligent gait-phase
detection system to identify pathological gait,” \emph{IEEE Trans. Inf. Technol.
	Biomed.}, vol. 14, no. 5, pp. 1173–1179, Sep. 2010.
\bibitem{biol1998} R. Durbin and S. Eddy and A. Krogh and G. Mitchison, Markov chains and hidden Markov models", in \emph{Biological sequence analysis}, UK: Cambridge University Press, 1998, pp. 46-79.
\bibitem{thes2011}G. Pfundstein, "Hidden Markov Models with Generalised Emission Distribution for Analysis of High-Dimensional, Non-Euclidean Data," M.S. thesis, Dept. Stat. and Gene Center, Munich Univ., Munich, Germany, 2011.
\bibitem{tool2001} Huseyin M. Ertunc and Kenneth A. Loparo and Hasan Ocak, "Tool wear condition monitoring in drilling operations using hidden Markov models (HMMs)", \emph{International Journal of Machine Tools and Manufacture}, vol. 41, no. 9, pp. 1363-1384, 2001, ISSN 0018-9456, DOI https://doi.org/10.1016/S0890-6955(00)00112-7. [Online]. Available:  http://www.sciencedirect.com/science/article/pii/S0890695500001127
\bibitem{effi2016} Mingxia Liu and Daoqiang Zhang, "Feature selection with effective distance", \emph{Neurocomputing}, vol. 215, no. Supplement C, pp. 100-109, 2016, note SI: Stereo Data, ISSN 0925-2312, DOI https://doi.org/10.1016/j.neucom.2015.07.155. [Online]. Available: http://www.sciencedirect.com/science/article/pii/S0925231216306336. 
\bibitem{twop2008} Ali Mahvash and Annie Ross, "Two-phase flow pattern identification using continuous hidden Markov model", \emph{International Journal of Multiphase Flow}, vol. 34, no. 3, pp. 303-311, 2008, ISSN 0301-9322, DOI https://doi.org/10.1016/j.ijmultiphaseflow.2007.08.006. [Online]. Available: http://www.sciencedirect.com/science/article/pii/S0301932207001462. 
\bibitem{newh2016} Saul Solorio-Fernandez and J. Ariel Carrasco-Ochoa and José Fco. Martínez-Trinidad, "A new hybrid filter–wrapper feature selection method for clustering based on ranking", \emph{Neurocomputing}, vol. 214, no. Supplement C, pp. 866-880, 2016, ISSN 0925-2312, DOI https://doi.org/10.1016/j.neucom.2016.07.026. [Online]. Available: http://www.sciencedirect.com/science/article/pii/S0925231216307718. 
\bibitem{hybr2011} Hui-Huang Hsu and Cheng-Wei Hsieh and Ming-Da Lu, "Hybrid feature selection by combining filters and wrappers", \emph{Expert Systems with Applications}, vol. 38, no. 7, pp. 8144-8150, 2011, ISSN 0957-4174, DOI https://doi.org/10.1016/j.eswa.2010.12.156. [Online]. Available: http://www.sciencedirect.com/science/article/pii/S0957417410015198. 
\bibitem{sima2013} Jeong-Su Han and Sang Wan Lee and Zeungnam Bien, "Feature subset selection using separability index matrix", \emph{Information Sciences}, vol. 223, no. Supplement C, pp. 102-118, 2013, ISSN 0020-0255, DOI https://doi.org/10.1016/j.ins.2012.09.042. [Online]. Available: http://www.sciencedirect.com/science/article/pii/S0020025512006354. 
\bibitem{maha2013} Zhao Yongli and Zhang Yungui and Tong Weiming and Chen Hongzhi, "An improved feature selection algorithm based on MAHALANOBIS distance for Network Intrusion Detection", \emph{PROCEEDINGS OF 2013 International Conference on Sensor Network Security Technology and Privacy Communication System}, pp. 69-73, 2013, DOI 10.1109/SNS-PCS.2013.6553837.
\bibitem{prtool} Dimension Reduction - Pattern Recognition Tools.[Online]. Available: http://37steps.com/prtools/examples/dimension-reduction/
\bibitem{lda2015} Youness Aliyari Ghassabeh and Frank Rudzicz and Hamid Abrishami Moghaddam, "Fast incremental LDA feature extraction", \emph{Pattern Recognition}, vol. 48, no. 6, pp. 1999-2012, 2015, ISSN 0031-3203, DOI https://doi.org/10.1016/j.patcog.2014.12.012. [Online]. Available: http://www.sciencedirect.com/science/article/pii/S0031320314005214.
\bibitem{dime2017} Miguel Simão and Pedro Neto and Olivier Gibaru, "Using data dimensionality reduction for recognition of incomplete dynamic gestures", \emph{Pattern Recognition Letters}, vol. 99, no. Supplement C, pp. 32-38, 2017, ISSN 0167-8655, DOI https://doi.org/10.1016/j.patrec.2017.01.003. [Online]. Available: http://www.sciencedirect.com/science/article/pii/S016786551730003X.
\bibitem{pendu2004} Timothy M. Griffin and Russell P. Main and Claire T. Farley, "Biomechanics of quadrupedal walking: how do four-legged animals achieve
inverted pendulum-like movements?", \emph{The Journal of Experimental Biology}, vol. 207, pp. 3545-3558, 2004, DOI 10.1242/jeb.01177.
\bibitem{mat-feat} Feature Selection - MathWorks.[Online]. Available: https://www.mathworks.com/help/stats/feature-selection.html
\bibitem{knnc} K-Nearest Neighbor Classifier - Pattern Recognition Tools.[Online]. Available: http://www.37steps.com/prhtml/prtools/knnc.html
\bibitem{hmmestimate} hmmestimate - MathWorks.[Online]. Available: https://www.mathworks.com/help/stats/hmmestimate.html
\bibitem{gmm2000} McLachlan, G., and D. Peel, Finite Mixture Models. Hoboken, NJ: \emph{John Wiley \& Sons}, Inc., 2000.
\bibitem{fitgmm} fitgmdist - MathWorks.[Online]. Available:  https://www.mathworks.com/help/stats/fitgmdist.html
\bibitem{AIC_ref} Ljung, L. System Identification: Theory for the User, Upper Saddle River, NJ, Prentice-Hall PTR, 1999
\bibitem{AIC} Akaike's Information Criterion for estimated model - MATLAB aic.[Online]. Available:  https://www.mathworks.com/help/ident/ref/aic.html
\bibitem{BIC} Bayes Information Criterion - MATLAB - MathWorks.[Online]. Available: https://www.mathworks.com/help/stats/gmdistribution.bic.html
\bibitem{matlab-hmm} Qiuqiang Kong, "matlab-hmm".[Online]. Available: https://github.com/qiuqiangkong/matlab-hmm
\bibitem{featsel} Trainable mapping for forward feature selection - Pattern Recognition Tools. [Online] . http://www.37steps.com/prhtml/prtools/featself.html
\bibitem{maha} Mahalanobis distance - Pattern Recognition Tools. [Online] http://www.37steps.com/prhtml/prtools/distmaha.html
\bibitem{pcam} Principal component analysis (PCA or MCA on overall covariance matrix) - Pattern Recognition Tools. [Online] http://www.37steps.com/prhtml/prtools/pcam.html
\bibitem{ldam} Optimal discrimination linear mapping (Fisher mapping, LDA) - Pattern Recognition Tools. [Online]. http://www.37steps.com/prhtml/prtools/fisherm.html
\bibitem{cvpartition} Create cross validation partition for data - MATLAB - MathWorks.[Online]. https://www.mathworks.com/help/stats/cvpartition.html
\bibitem{gendat} Random sampling of datasets for training and testing - Pattern Recognition Tools. [Online]. http://www.37steps.com/prhtml/prtools/gendat.html
\bibitem{prtool-home} PRTools - A Matlab toolbox for pattern recognition.[Online]. http://prtools.org/
\bibitem{matlab-stats} Statistics and Machine Learning Toolbox - MATLAB - MathWorks.[Online]. https://www.mathworks.com/products/statistics.html
\end{thebibliography}
\appendix
\chapter{Additional Files and Schematics}

\section{Implementation and experiment code} \label{apdix-repo}
The implementation can be accessed from this github repository \href{https://github.com/h-kouame/gait-sequence-modelling-and-estimation-v2}{gait-sequence-modelling-and-estimation-v2}
\chapter{Addenda}

\section{Ethics Forms}
\begin{figure}[ht!]
	\includegraphics[width=1.5\textwidth, angle=90]{Figures/ethics-form}
%	\caption{The effect of training datasize on the log likelihood}
%	\label{fig:size-log}
\end{figure}
}
\end{document}
