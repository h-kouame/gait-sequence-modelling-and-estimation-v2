 % Copyright 2008 by Till Tantau and Mark Wibrow
%
% This file may be distributed and/or modified
%
% 1. under the LaTeX Project Public License and/or
% 2. under the GNU Free Documentation License.
%
% See the file doc/generic/pgf/licenses/LICENSE for more details.

\section{Decorations}
\label{section-base-decorations}


\begin{pgfmodule}{decorations}
  The commands for creating decorations are defined in this
  module, so you need to load this module to use decorations. This
  module is automatically loaded by the different decoration
  libraries.
\end{pgfmodule}


\subsection{Overview}

Decorations are a general way of creating graphics by ``moving along''
a path and, while doing so, either drawing something or constructing a
new path. This could be as simple as extending a path with a
``zigzagged'' line\ldots

\begin{codeexample}[]
\tikz \draw decorate[decoration=zigzag] {(0,0) -- (3,0)};
\end{codeexample}
\ldots but could also be as complex as typesetting text along a path:
{\catcode`\|12
\begin{codeexample}[]
\tikz \path decorate [decoration={text along path,
     text={Some text along a path}}]
   { (0,2) .. controls (2,2) and (1,0) .. (3,0) };
\end{codeexample}
}

The workflow for using decorations is the following:
\begin{enumerate}
\item You define a decoration using the |\pgfdeclaredecoration|
  command. Different useful decorations are already declared in
  libraries like |decorations.shapes|.
\item You use normal path construction commands like |\pgfpathlineto|
  to construct a path. Let us call this path the
  \emph{to-be-decorated} path.
\item You place the path construction commands inside the environment
  |{pgfdecoration}|. This environment takes the name of a previously
  declared decoration as a parameter. It will then start ``walking
  along'' the to-be-decorated path. As it does this, a special finite
  automaton called a \emph{decoration automaton} produces new path commands as its output (or even other outputs). These
  outputs replace the to-be-decorated path; indeed, after the
  to-be-decorated path has been fully walked along it is thrown away,
  only the output of the automaton persists.
\end{enumerate}

In the present section the process of how decoration automata work is
explained first. Then the command(s) for declaring decoration automata
and for using them are covered.



\subsection{Decoration Automata}

Decoration automata (and the closely related meta-decoration automata)
are a general concept for creating graphics ``along paths.'' For
straight lines, this idea was first proposed by Till Tantau in an
earlier version of \pgfname, the idea to extend this to arbitrary path
was proposed and implemented by Mark Wibrow. Further versatility is
provided by ``meta-decorations''. These are automata that decorate a
path with decorations.

In the present subsection the different ideas underlying decoration
automata are presented.



\subsubsection{The Different Paths}

In order to prevent confusion with different types of paths, such
as those that are extended, those that are decorated and those that
are created, the following conventions will be used:

\begin{itemize}
\item
  The \emph{preexisting} path refers to the current path in existence
  before a decoration environment. (Possibly this path has been
  created by another decoration used earlier, but we will still call
  this path the preexisting path also in this case.)
\item
  The \emph{input} path refers to the to-be-decorated path that the
  decoration automaton moves along. The input path may consist of many
  line and curve input segments (for example, a circle or an ellipse
  consists of four curves). It is specified inside the decoration
  environment.
\item
  The \emph{output} path refers to the path that the decoration
  creates. Depending on the decoration, this path may or may not be
  empty (a decoration can also choose to use side-effects instead of
  producing an output path). The input path is always consumed by the
  decoration automaton, that is, it is no longer available in any way
  after the decoration automaton has finished.
\end{itemize}

The effect of a decoration environment is the following: The input
path, which is specified inside the environment, is constructed and
stored. This process does not alter the preexisting path in any
way. Then the decoration automaton is started (as described later) and
it produces an output path (possibly empty). Whenever part of the
output path is produced, it is concatenated with the preexisting
path. After the environment, the current path will equal the original
preexisting path followed by the output path.

It is permissible that a decoration issues a |\pgfusepath|
command. As usual, this causes the current path to be
filled or stroked or some other action to be taken and the current
path is set to the empty path. As described above, when the decoration
automaton starts, the current path is the preexisting path and as the
automaton progresses, the current path is constantly being extended by
the output path. This means that first time a |\pgfusepath| command is
used on a decoration, the preexisting path is part of the path this
command operates on; in subsequent calls only the part of the output
path constructed since the last |\pgfusepath| command will be used.

You can use this mechanism to stroke or fill different parts of the
output path in different colors, line widths, fills and shades; all
within the same decoration. Alternatively, a decoration can choose to
produce no output path at all: the |text| decoration simply typesets
text along a path.


\subsubsection{Segments and States}

The most common use of a decoration is to ``repeat something along a
path'' (for example, the |zigzag| decoration
repeats \tikz\draw decorate[decoration=zigzag]
{(0,0)--(\pgfdecorationsegmentlength,0)}; along a path). However, it
not necessarily the case that only one thing is repeated: a decoration
can consist of different parts, or \emph{segments}, repeated in a
particular order.

When you declare a decoration, you provide a description
of how their different segments will be rendered. The description of
each segment should be given in a way as if the ``$x$-axis'' of the
segment is the tangent to the path at a particular point,
and that point is the origin of the segment.
Thus, for example, the segment of the |zigzag| decoration might be
defined using the following code:
\begin{codeexample}[code only]
\pgfpathlineto{\pgfpoint{5pt}{5pt}}
\pgfpathlineto{\pgfpoint{15pt}{-5pt}}
\pgfpathlineto{\pgfpoint{20pt}{0pt}}
\end{codeexample}

\pgfname\ will ensure that an appropriate coordinate transformation
is in place when the segment is rendered such that
the segment actually points in the right direction. Also,
subsequent segments will be transformed such that they are
``further along the path'' toward the end of the path.
All transformations are set up automatically.

Note that we did not use a |\pgfpathmoveto{\pgfpointorigin}| at the
beginning of the segment code. Doing so would subdivide the path into
numerous subpaths. Rather, we assume that the previous segment caused
the current point to be at the origin.

The width of a segment can (and must) be specified
explicitly. \pgfname\ will use this width to find out the start point
of the next segment and the correct rotation. The width the you
provide need not be the ``real'' width of the segment, which allows
decoration segments to overlap or to be spaced far apart.

The |zigzag| decoration only has one segment that is repeated again and
again. However, we might also like to have \emph{different} segments
and use rules to describe which segment should be used where. For
example, we might have special segments at the start and at the end.

Decorations use a mechanism known in theoretical in computer science
as \emph{finite state automata} to describe which segment is used at a
particular point. The idea is the following: For the first segment we
start in a special \emph{state} called the \emph{initial state}. In
this state, and also in all other states later, \pgfname\ first
computes how much space is left on the input path. That is, \pgfname\ keeps
track of the distance to the end of the input path. Attached to each state
there is a set of rules of the following form: ``If the remaining
distance on the input path is less than $x$, switch to state~$q$.''
\pgfname\ checks for each of these rules whether it applies and, if
so, immediately switches to state~$q$.

Only if none of the rules tell us to switch to another
state, \pgfname\ will execute the state's code. This code will
(typically) add a segment to the output path. In addition to the rules
there is also a width parameter attached to each state. \pgfname\ then
translates the coordinate system by this width and reduces the
remaining distance on the input path. Then, \pgfname\ either stays in
the current state or switches to another state, depending on yet
another property attached of the state.

The whole process stops when a special state called |final| is
reached. The segment of this state is immediately added to the output
path (it is often empty, though) and the process ends.




\subsection{Declaring Decorations}

The following command is used to declare a decoration. Essentially,
this command describes the decoration automaton.


\begin{command}{\pgfdeclaredecoration\marg{name}\marg{initial
      state}\marg{states}}
  This command declares a new decoration called \meta{name}. The
  \meta{states} argument contains a description of the decoration
  automaton's states and the transitions between them. The
  \meta{initial state} is the state in which the automaton starts.

  When the automaton is later applied to an input path, it keeps track
  of a certain position on the input path. This current point
  will ``travel along the path,'' each time being moved along by a
  certain distance. This will also work if the path is not a straight
  line. That is, it is permissible that the path curves are veers at a
  sharp angle.  It is also permissible that while traveling along the
  input path, the current input segment ends and a new input segment starts. In this
  case, the remaining distance on the first input segment is subtracted
  from the \meta{dimension} and then we travel along the second
  input segment for the remaining distance. This input segment may also end
  early, in which case we travel along the next input segment, and so
  on. Note that it cannot happen that we travel past the end of the
  input path since this would have caused an immediate switch to
  the |final| state.

  Note that the computation of the path lengths has only a low
  accuracy because of \TeX's small math capabilities. Do not
  expect high accuracy alignments when using decorations (unless the
  input path consists only of horizontal and vertical lines).

  The \meta{states} argument should consist of |\state| commands, one
  for each state of the decoration automaton. The |\state| command is
  defined only when the \meta{states} argument is executed.

  \begin{command}{\state\marg{name}\oarg{options}\marg{code}}
    This command declares a new state inside the current decoration
    automaton. The state is named \meta{name}.

    When the decoration automaton is in state \meta{name}, the following things
    happen:
    \begin{enumerate}
    \item
      The \meta{options} are parsed. This may lead to a
      state switch, see below. When this happens, the following steps are not
      executed. The \meta{options} are executed one after the other in
      the given order. If an option causes a state switch, the switch
      is immediate, even if later options might cause a different
      state switch.
    \item
      The \meta{code} is executed in a \TeX-group with the current
      transformation matrix set up in such a way that the origin is on
      the input path at the current point (the point at the distance
      traveled up to now) and the coordinate system is rotated in
      such a way that the positive $x$-axis points in the direction of
      the tangent to the input path at the current point, while the
      positive $y$-axis points to the left of this tangent.

      As described earlier, the \meta{code} can have two different
      effects: If it just contains path construction commands, the
      decoration will produce an output path, that is, it extends the
      preexisting path. Here is an example:

\begin{codeexample}[]
\pgfdeclaredecoration{example}{initial}
{
  \state{initial}[width=10pt]
  {
    \pgfpathlineto{\pgfpoint{0pt}{5pt}}
    \pgfpathlineto{\pgfpoint{5pt}{5pt}}
    \pgfpathlineto{\pgfpoint{5pt}{-5pt}}
    \pgfpathlineto{\pgfpoint{10pt}{-5pt}}
    \pgfpathlineto{\pgfpoint{10pt}{0pt}}
  }
  \state{final}
  {
    \pgfpathlineto{\pgfpointdecoratedpathlast}
  }
}
\tikz[decoration=example]
{
  \draw [decorate]     (0,0) -- (3,0);
  \draw [red,decorate] (0,0) to [out=45,in=135] (3,0);
}
\end{codeexample}

    Alternatively, the \meta{code} can also contain the
    |\pgfusepath| command. This will use the path in the usual manner,
    where ``the path'' is the preexisting path plus a part of the
    output path for the first invocation and the different parts of
    the rest of the output path for the following invocation. Here is
    an example:

\begin{codeexample}[]
\pgfdeclaredecoration{stars}{initial}{
  \state{initial}[width=15pt]
  {
    \pgfmathparse{round(rnd*100)}
    \pgfsetfillcolor{yellow!\pgfmathresult!orange}
    \pgfsetstrokecolor{yellow!\pgfmathresult!red}
    \pgfnode{star}{center}{}{}{\pgfusepath{stroke,fill}}
  }
  \state{final}
  {
    \pgfpathmoveto{\pgfpointdecoratedpathlast}
  }
}
\tikz\path[decorate, decoration=stars, star point ratio=2, star points=5,
           inner sep=0, minimum size=rnd*10pt+2pt]
  (0,0) .. controls (0,2)  and (3,2)  .. (3,0)
        .. controls (3,-3) and (0,0)  .. (0,-3)
        .. controls (0,-5) and (3,-5) .. (3,-3);
\end{codeexample}

    \item
      After the \meta{code} has been executed (possibly more than
      once, if the |repeat state| option is used), the state switches to
      whatever state has been specified inside the \meta{options}
      using the |next state| option. If no |next state| has been
      specified, the state stays the same.
    \end{enumerate}

    The \meta{options} are executed with the key path set to
    |/pgf/decoration automaton|. The following keys are defined:
    \begin{key}{/pgf/decoration automaton/switch if less than=\meta{dimension}\texttt{ to }\meta{new state}}
      When this key is encountered, \pgfname\ checks whether the
      remaining distance to the end of the input path is less than
      \meta{dimension}. If so, an immediate state switch to \meta{new
        state} occurs.
    \end{key}
    \begin{key}{/pgf/decoration automaton/switch if input segment less than=\\\meta{dimension}\texttt{ to }\meta{new state}}
      When this key is encountered, \pgfname\ checks whether the
      remaining distance to the end of the current input segment of the
      input path is less than \meta{dimension}. If so, an immediate
      state switch to \meta{new state} occurs.
    \end{key}
    \begin{key}{/pgf/decoration automaton/width=\meta{dimension}}
      First, this option causes an immediate switch to the
      state |final| if the remaining distance on the input path is
      less than \meta{dimension}. The effect is the same as if you had
      said |switch if less than=|\meta{dimension}| to final| just
      before the |width| option.

      If no switch occurs, this option tells \pgfname\ the width of
      the segment. The current point will travel along the input path
      (as described earlier)   by this distance.
    \end{key}
    \begin{key}{/pgf/decoration automaton/repeat state=\meta{repetitions} (initially 0)}
      Tells \pgfname\ how long the automaton stays ``normally'' in the
      current state. This count is reset to \meta{repetitions} each
      time one of the |switch if| keys causes a state switch. If no
      state switches occur, the \meta{code} is executed and the
      repetition counter is decreased. Then, there is once more a
      chance of a state change caused by any of the \meta{options}. If
      no repetition occurs, the \meta{code} is executed
      once more and the counter is decreased once more. When the
      counter reaches zero, the \meta{code} is executed once more,
      but, then, a different state is entered, as specified by the
      |next state| option.

      Note, that the maximum number of times the state will be executed
      is $\meta{repetitions}+1$.
    \end{key}
    \begin{key}{/pgf/decoration automaton/next state=\meta{new state}}
      After the \meta{code} for state has been executed for the last
      time, a state switch to \meta{new state} is performed. If this
      option is not given, the next state is the same as the current state.
    \end{key}

    \begin{key}{/pgf/decoration automaton/if input segment is closepath=\meta{options}}
      This key checks whether the current input segment is a closepath
      operation. If so, the \meta{options} get executed; otherwise
      nothing happens. You can use this option to handle a closepath
      in some special way, for instance, switching to a new state in
      which |\pgfpathclose| is executed.
    \end{key}

    \begin{key}{/pgf/decoration automaton/auto end on length=\meta{dimension}}
      This key is just included for convenience, it does nothing that
      cannot be achieved using the previous options. The effect is the
      following: If the remaining input path's length is at most
      \meta{dimension}, the decorated path is ended with a straight
      line to the end of the input path and, possibly, it is closed,
      namely if the input path ended with a closepath
      operation. Otherwise, it is checked whether the current input
      segment is a closepath segment and whether the remaining
      distance on the current input segment is at most
      \meta{distance}. If so, then a closepath operation is used to
      close the decorated path and the automaton continues with the
      next subpath, remaining in the current state.

      In all other cases, nothing happens.
    \end{key}

    \begin{key}{/pgf/decoration automaton/auto corner on length=\meta{dimension}}
      This key has the following effect: Firstly, in case the \TeX-if
      |\ifpgfdecoratepathhascorners| is false, nothing
      happens. Otherwise, it is tested whether the remaining distance
      on the current input segment is at most \meta{dimension}. If so,
      a |lineto| operation is used to reach the end of this input
      segment and the automaton continues with the next input segment,
      but remains in the current state.

      The main idea behind this option is to avoid having decoration
      segments ``overshoot'' past a corner.
    \end{key}

    You may sometimes wish to do computations outside the
    transformational \TeX-group of the current segment,
    so that these results of these computations are available in the
    next state. For this, the following two options are useful:

    \begin{key}{/pgf/decoration automaton/persistent precomputation=\meta{precode}}
      If the \meta{code} of the state is executed, the \meta{precode} is
      executed first and it is executed outside the \TeX-group of the
      \meta{code}. Note that when the \meta{precode} is executed, the
      transformation matrix is not set up.
    \end{key}

    \begin{key}{/pgf/decoration automaton/persistent postcomputation=\meta{postcode}}
      Works like the |persistent precomputation| option, only the
      \meta{postcode} is executed after (and also outside) the
      \TeX-group of the main \meta{code}.
    \end{key}

    There are a number of macros and dimensions which may be useful
    inside a decoration automaton. The following macros are available:

    \begin{command}{\pgfdecoratedpathlength}
      The length of the input path. If the input path consists of
      several input segments, this number is the sum of the lengths of the
      input segments.
    \end{command}

    \begin{command}{\pgfdecoratedinputsegmentlength}
      The length of the current input segment of the input path. ``Current
      input segment''  refers to the input segment on which the current point
      lies.
    \end{command}
		
    \begin{command}{\pgfpointdecoratedpathlast}
      The final point of the input path.
    \end{command}

    \begin{command}{\pgfpointdecoratedinputsegmentlast}
      The final point of the current input segment of the input path.
    \end{command}
		
    \begin{command}{\pgfdecoratedangle}
      The angle of the tangent to the decorated path at the \emph{origin}
      of the current segment. The transformation matrix applied at
      the beginning of a state includes a rotation equivalent to
      this angle.
    \end{command}
		
    The following \TeX\ dimension registers are also available inside the
    automaton:

    \begin{command}{\pgfdecoratedremainingdistance}
      The remaining distance on the input path.
    \end{command}
		
    \begin{command}{\pgfdecoratedcompleteddistance}
      The completed distance on the input path.
    \end{command}

    \begin{command}{\pgfdecoratedinputsegmentremainingdistance}
      The remaining distance on the current input segment of the input path.
    \end{command}

    \begin{command}{\pgfdecoratedinputsegmentcompleteddistance}
      The completed distance on the current input segment of the input path.
    \end{command}

    Further keys and macros are defined and used by the decoration
    libraries, see Section~\ref{section-library-decorations}.

    The following example shows how these options can be used:
\begin{codeexample}[]
\pgfdeclaredecoration{complicated example decoration}{initial}
{
  \state{initial}[width=5pt,next state=up]
  { \pgfpathlineto{\pgfpoint{5pt}{0pt}} }

  \state{up}[width=5pt,next state=down]
  {
    \ifdim\pgfdecoratedremainingdistance>\pgfdecoratedcompleteddistance
      % Growing
      \pgfpathlineto{\pgfpoint{0pt}{\pgfdecoratedcompleteddistance}}
      \pgfpathlineto{\pgfpoint{5pt}{\pgfdecoratedcompleteddistance}}
      \pgfpathlineto{\pgfpoint{5pt}{0pt}}
    \else
      % Shrinking
      \pgfpathlineto{\pgfpoint{0pt}{\pgfdecoratedremainingdistance}}
      \pgfpathlineto{\pgfpoint{5pt}{\pgfdecoratedremainingdistance}}
      \pgfpathlineto{\pgfpoint{5pt}{0pt}}
    \fi%
  }
  \state{down}[width=5pt,next state=up]
  {
    \ifdim\pgfdecoratedremainingdistance>\pgfdecoratedcompleteddistance
      % Growing
      \pgfpathlineto{\pgfpoint{0pt}{-\pgfdecoratedcompleteddistance}}
      \pgfpathlineto{\pgfpoint{5pt}{-\pgfdecoratedcompleteddistance}}
      \pgfpathlineto{\pgfpoint{5pt}{0pt}}
    \else
      % Shrinking
      \pgfpathlineto{\pgfpoint{0pt}{-\pgfdecoratedremainingdistance}}
      \pgfpathlineto{\pgfpoint{5pt}{-\pgfdecoratedremainingdistance}}
      \pgfpathlineto{\pgfpoint{5pt}{0pt}}
    \fi%
  }
  \state{final}
  {
    \pgfpathlineto{\pgfpointdecoratedpathlast}
  }
}
\begin{tikzpicture}[decoration=complicated example decoration]
  \draw decorate{ (0,0) -- (3,0)};
  \fill [red!50,rounded corners=2pt]
    decorate {(.5,-2) -- ++(2.5,-2.5)} -- (3,-5) -| (0,-2) -- cycle;
\end{tikzpicture}
\end{codeexample}
  \end{command}
\end{command}



\subsubsection{Predefined Decorations}

The three decorations |moveto|, |lineto|, and |curveto| are predefined
and ``always available.'' They are mostly useful in conjunction with
meta-decorations. They are documented in
Section~\ref{section-library-decorations} alongside the other
decorations.



\subsection{Using Decorations}

Once a decoration has been declared, it can be used.

\begin{environment}{{pgfdecoration}\marg{decoration list}}
  The \meta{environment contents} should contain commands for creating
  an path. This path is the basis for the \emph{input paths}
  for the decorations in the \meta{decoration list}. In detail, the
  following happens:
  \begin{enumerate}
  \item
    The preexisting unused path is saved.
  \item
    The path commands specified in \meta{environment contents} are
    executed and this resulting path is saved. The path is then
    divided into different \emph{input paths} as follows:
    The format for each item in \marg{decoration list} is
    \begin{quote}
      \marg{decoration}\marg{length}\opt{\marg{before code}\marg{after code}}
    \end{quote}
    The \meta{before code} and the \meta{after code} are optional. The
    input path is divided into input paths as follows: The first input
    path consists of the first lines of the path specified in the
    \meta{environment contents} until the \meta{length}  of the first
    element of the \meta{decoration list} is reached. If this length
    is reached in the middle of a line, the line is broken up at this
    exact position. Then the second input path has the \meta{length}
    of the second element in the \meta{decoration list} and consists
    of the lines making up the following \meta{length} part of the
    path in the \meta{environment contents}, and so on.

    If the lengths in the \meta{decoration list}
    do not add up to the total length of the path in the
    \meta{environment contents}, either some  decorations are dropped
    (if their lengths add up to more than the length of the
    \meta{environment contents}) or
    the input path is not fully used (if their lengths  add up to less).
  \item
    The preexisting path is reinstalled.
  \item
    The decoration automata move along the input paths, thus creating
    (and  possibly using) the output paths. These output paths extend
     the current path (unless they are used).
  \end{enumerate}
	
  Some important points should be noted regarding the use of this
  environment:

  \begin{itemize}
  \item
    If \meta{environment contents} does not begin with
    |\pgfpathmoveto|, the last known point on the preexisting path is
    assumed as the starting point.
  \item
    All except the last of any sequence of consecutive move-to commands
    in \meta{environment contents} are discarded.
  \item
    Any move-to commands at the end of \meta{environment contents} are
    ignored.
  \item
    Any close-path commands on the input path are interpreted as
    straight lines.
    Internally, something a little more complicated is going on,
    however, a closed path on the input path has no effect on the
    output path, other than causing the automaton to travel in a
    straight line towards the location of the last move-to command on
    the input path.
  \item
    Although tangent computations for the input path work with the
    last point on the preexisting path, no automatic move-to
    operations are issued for the output path.
    If an output path starts with a line-to or curve-to when the
    existing path is empty, an appropriate move-to command should be
    inserted before the decoration starts.
  \item
    If a decoration uses its own path, the first time this happens the
    preexisting path is part of the path that is used at this point.
  \end{itemize}

Before the automata start to "work on" their respective inputs paths,
  \meta{before code} is
  executed. After the decoration automaton has finished, \meta{after
    code} is executed.

\begin{codeexample}[]
\begin{tikzpicture}[decoration={segment length=5pt}]
  \draw [help lines] grid (3,2);
  \begin{pgfdecoration}{{curveto}{1cm},{zigzag}{2cm},{curveto}{1cm}}
    \pgfpathmoveto{\pgfpointorigin}
    \pgfpathcurveto
      {\pgfpoint{0cm}{2cm}}{\pgfpoint{3cm}{2cm}}{\pgfpoint{3cm}{0cm}}
  \end{pgfdecoration}
\pgfusepath{stroke}
\end{tikzpicture}
\end{codeexample}

  When the lengths are evaluated, the dimension
  |\pgfdecoratedremainingdistance| holds the remaining distance on
  the entire decorated path, and |\pgfdecoratedpathlength| holds the
  total length of the path. Thus, it is possible to specify lengths
  like |\pgfdecoratedpathlength/3|.

\begin{codeexample}[]
\begin{tikzpicture}[decoration={segment length=5pt}]
  \draw [help lines] grid (3,2);
  \begin{pgfdecoration}{
      {curveto}{\pgfdecoratedpathlength/3},
      {zigzag}{\pgfdecoratedpathlength/3},
      {curveto}{\pgfdecoratedremainingdistance}
    }
    \pgfpathmoveto{\pgfpointorigin}
    \pgfpathcurveto
      {\pgfpoint{0cm}{2cm}}{\pgfpoint{3cm}{2cm}}{\pgfpoint{3cm}{0cm}}
  \end{pgfdecoration}
  \pgfusepath{stroke}
\end{tikzpicture}
\end{codeexample}

  When \meta{before code} is executed, the following macro is useful:
  \begin{command}{\pgfpointdecoratedpathfirst}
    Returns the point corresponding to the start of the current
    input path.
  \end{command}
  When \meta{after code} is executed, the following macro can be used:
  \begin{command}{\pgfpointdecoratedpathlast}
    Returns the point corresponding to the end of the current input
    path.
  \end{command}
  This means that if decorations do not use their own path, it is
  possible to do something with them and continue from the
  correct place.
	
\begin{codeexample}[]
\begin{tikzpicture}
  \draw [help lines] grid (3,2);
  \begin{pgfdecoration}{
      {curveto}{\pgfdecoratedpathlength/3}
      {}
      {
        \pgfusepath{stroke}
      },
      {zigzag}{\pgfdecoratedpathlength/3}
      {
        \pgfpathmoveto{\pgfpointdecoratedpathfirst}
        \pgfdecorationsegmentlength=5pt
      }
      {
        \pgfsetstrokecolor{red}
        \pgfusepath{stroke}
        \pgfpathmoveto{\pgfpointdecoratedpathlast}
        \pgfsetstrokecolor{black}
      },
      {curveto}{\pgfdecoratedremainingdistance}
    }
    \pgfpathmoveto{\pgfpointorigin}
    \pgfpathcurveto
      {\pgfpoint{0cm}{2cm}}{\pgfpoint{3cm}{2cm}}{\pgfpoint{3cm}{0cm}}
  \end{pgfdecoration}
  \pgfusepath{stroke}
\end{tikzpicture}
\end{codeexample}
	
  After the |{decoration}| environment has finished, the following
  macros are available:
	
  \begin{command}{\pgfdecorateexistingpath}
    The preexisting path before the environment was entered.
  \end{command}
	
  \begin{command}{\pgfdecoratedpath}
    The (total) input path (that is, the path created by the environment contents).
  \end{command}
	
  \begin{command}{\pgfdecorationpath}
    The output path. If the path is used, this macro contains only the
    last unused part of the output path.
  \end{command}
	
  \begin{command}{\pgfpointdecoratedpathlast}
    The final point of the input path.
  \end{command}

  \begin{command}{\pgfpointdecorationpathlast}
    The final point of the output path.
  \end{command}

  The following style is executed each time a decoration is used. You
  may use it to set up default options for decorations.
  \begin{stylekey}{/pgf/every decoration (initially \normalfont empty)}
    This style is executed for every decoration.
  \end{stylekey}
\end{environment}

\begin{plainenvironment}{{pgfdecoration}\marg{name}}
  The plain-\TeX{} version of the |{pgfdecorate}| environment.      % TODOsp: `pgfdecorate' right or `pgfdecoration'? really no idea
\end{plainenvironment}

\begin{contextenvironment}{{pgfdecoration}\marg{name}}
  The Con\TeX t version of the |{pgfdecoration}| environment.
\end{contextenvironment}

For convenience, the following macros provide a ``shorthand''
for decorations (internally, they all use the |{pgfdecoration}|
environment).

\begin{command}{\pgfdecoratepath\marg{name}\marg{path commands}}
  Decorate the path described by \meta{path commands} with the
  decoration \meta{name}. This is equivalent to
\begin{codeexample}[code only]
\pgfdecorate{{name}{\pgfdecoratedpathlength}
             {\pgfdecoratebeforecode}{\pgfdecorateaftercode}}
  // the path commands.
\endpgfdecorate
\end{codeexample}
\end{command}

\begin{command}{\pgfdecoratecurrentpath\marg{name}}
  Decorate the preexisting path with the decoration \meta{name}.
\end{command}

Both the above commands use the current definitions of the following
macros:

\begin{command}{\pgfdecoratebeforecode}
  Code executed as \meta{before code}, see the description of
  |\pgfdecorate|.
\end{command}

\begin{command}{\pgfdecorateaftercode}
  Code executed as \meta{after code}, see the description of
  |\pgfdecorate|.
\end{command}

It may sometimes be useful to add an additional transformation
for each segment of a decoration. The following command allows
you to define such a ``last minute transformation.''

\begin{command}{\pgfsetdecorationsegmenttransformation\marg{code}}
  The \meta{code} will be executed at the very beginning of each
  segment. Note when applying multiple decorations, this will
  be reset between decorations, so it needs to be specified for
  each segment.

\begin{codeexample}[]
\begin{tikzpicture}
  \draw [help lines] grid (3,2);
  \begin{pgfdecoration}{
      {curveto}{\pgfdecoratedpathlength/3},
      {zigzag}{\pgfdecoratedpathlength/3}
      {
        \pgfdecorationsegmentlength=5pt
        \pgfsetdecorationsegmenttransformation{\pgftransformyshift{.5cm}}
      },
      {curveto}{\pgfdecoratedremainingdistance}
    }
    \pgfpathmoveto{\pgfpointorigin}
    \pgfpathcurveto
      {\pgfpoint{0cm}{2cm}}{\pgfpoint{3cm}{2cm}}{\pgfpoint{3cm}{0cm}}
  \end{pgfdecoration}
  \pgfusepath{stroke}
\end{tikzpicture}
\end{codeexample}
\end{command}




\subsection{Meta-Decorations}

\label{section-base-meta-decorations}

A meta-decoration provides an alternative way to decorate a path with
multiple decorations. It is, in essence, an automaton that decorates
an input path with decoration automatons. In general, however, the end
effect is still that a path is decorated with other paths, and the input
path should be thought of as being divided into sub-input-paths, each with
their own decoration. Like ordinary decorations, a meta-decoration must be declared before it can be used.

\subsubsection{Declaring Meta-Decorations}

\begin{command}{\pgfdeclaremetadecorate\marg{name}\marg{initial state}\marg{states}}

  This command declares a new meta-decoration called \meta{name}. The
  \meta{states} argument contains a description of the meta-decoration
  automaton's states and the transitions between them. The
  \meta{initial state} is the state in which the automaton starts.

  The |\state| command is similar to the one found in
  decoration declarations, and takes the same form:

  \begin{command}{\state\marg{name}\oarg{options}\marg{code}}
    Declares the state \meta{name} inside the current meta-decoration
    automaton. Unlike decorations, states in meta-decorations are not
    executed within a group, which makes the persistent computation
    options superfluous. Consider using an initial state with
    |width=0pt| to do precalculations that could speed the execution
    of the meta-decoration.

    The \meta{options} are executed with the key path set to
    |/pgf/meta-decorations automaton/|, and the following keys are defined for
    this path:

    \begin{key}{/pgf/meta-decoration automaton/switch if less than=\meta{dimension}| to |\meta{new state}}
      This causes \pgfname\ to check whether the
      remaining distance to the end of the input path is less than
      \meta{dimension}, and, if so, to immediately switch to the state
      \meta{new state}. When this key is evaluated, the macro
      |\pgfmetadecoratedpathlength| will be defined as the total length of
      the decoration path, allowing for values such as
      |\pgfmetadecoratedpathlength/8|.
    \end{key}

    \begin{key}{/pgf/meta-decoration automaton/width=\meta{dimension}}
      As always, this option will cause an immediate switch to the
      state |final| if the remaining distance on the input path is less than
      \meta{dimension}.

      Otherwise, this option tells \pgfname\ the width of the
      ``meta-segment'', that is, the length of the sub-input-path
      which the decoration automaton specified  in \meta{code} will decorate.
    \end{key}

    \begin{key}{/pgf/meta-decoration automaton/next state=\meta{new state}}
      After the code for a state has been executed, a state switch to
      \meta{new state} is performed. If this option is not given, the
      next state is the same as the current state.
    \end{key}

    The code in \meta{code} is quite different from the code in a
    decoration state. In almost all cases only the following three
    macros will be required:

    \begin{command}{\decoration\marg{name}}
      This sets the decoration for the current state to \meta{name}.
      If this command is omitted, the |moveto| decoration will be
      used.
    \end{command}

    \begin{command}{\beforedecoration\marg{before code}}
      Defines \meta{before code} as (typically) \pgfname{} commands to be
      executed before the decoration is applied to the current segment.
      This command can be omitted.
      If you wish to set up some decoration specific parameters
      such as segment length, or segment amplitude, then they
      can be set in \meta{before code}.
    \end{command}

    \begin{command}{\afterdecoration\marg{after code}}
      Defines \meta{after code} as commands to be executed after the
      decoration has been applied to the current segment.
      This command can be omitted.
    \end{command}

    There are some macros that may be useful when creating
    meta-decorations (note that they are all macros):

    \begin{command}{\pgfpointmetadecoratedpathfirst}
      When the \meta{before code} is executed,
      this macro stores the first point on the current
      sub-input-path.
    \end{command}

    \begin{command}{\pgfpointmetadecoratedpathlast}
      When the \meta{after code} is executed,
      this macro stores the last point on the current
      sub-input-path.
    \end{command}

    \begin{command}{\pgfmetadecoratedpathlength}
      The entire length of the entire input path.
    \end{command}

    \begin{command}{\pgfmetadecoratedcompleteddistance}
      The completed distance on the entire input path.
    \end{command}

    \begin{command}{\pgfmetadecoratedremainingdistance}
      The remaining distance on the entire input path.
    \end{command}

    \begin{command}{\pgfmetadecoratedinputsegmentcompleteddistance}
      The completed distance on the current input segment of the entire
      input path.
    \end{command}

    \begin{command}{\pgfmetadecoratedinputsegmentremainingdistance}
      The remaining distance on the current input segment of the entire
      input path.
    \end{command}
  \end{command}

  Here is a complete example of a meta-decoration:

\begin{codeexample}[]
\pgfdeclaremetadecoration{arrows}{initial}{
  \state{initial}[width=0pt, next state=arrow]
  {	
    \pgfmathdivide{100}{\pgfmetadecoratedpathlength}
    \let\factor\pgfmathresult
    \pgfsetlinewidth{1pt}
    \pgfset{/pgf/decoration/segment length=4pt}
  }
  \state{arrow}[
    switch if less than=\pgfmetadecorationsegmentlength to final,
    width=\pgfmetadecorationsegmentlength/3,
    next state=zigzag]
  {
    \decoration{curveto}
    \beforedecoration
    {
      \pgfmathparse{\pgfmetadecoratedcompleteddistance*\factor}
      \pgfsetcolor{red!\pgfmathresult!yellow}
      \pgfpathmoveto{\pgfpointmetadecoratedpathfirst}
    }	
  }
  \state{zigzag}[width=\pgfmetadecorationsegmentlength/3, next state=end arrow]
  {
  	\decoration{zigzag}
  }
  \state{end arrow}[width=\pgfmetadecorationsegmentlength/3, next state=move]
  {
    \decoration{curveto}
    \beforedecoration{\pgfpathmoveto{\pgfpointmetadecoratedpathfirst}}
    \afterdecoration
    {	
      \pgfsetarrowsend{to}
      \pgfusepath{stroke}	
    }
  }
  \state{move}[width=\pgfmetadecorationsegmentlength/2, next state=arrow]{}
  \state{final}{}
}

\tikz\draw[decorate,decoration={arrows,meta-segment length=2cm}]
  (0,0) .. controls (0,2)   and (3,2)   .. (3,0)
        .. controls (3,-2)  and (0,-2)  .. (0,-4)
        .. controls (0,-6)  and (3,-6)  .. (3,-8)
        .. controls (3,-10) and (0,-10) .. (0,-8);
\end{codeexample}

\end{command}


\subsubsection{Predefined Meta-decorations}

There are no predefined meta-decorations loaded with \pgfname{}.


\subsubsection{Using Meta-Decorations}

Using meta-decorations is ``simpler'' than using decorations, because
you can only use one meta-decoration per path.

\begin{environment}{{pgfmetadecoration}\marg{name}}
  This environment decorates the input path described in
  \meta{environment contents}, with the	meta-decoration \meta{name}.
\end{environment}

\begin{plainenvironment}{{pgfmetadecoration}\marg{name}}
  The plain \TeX{} version of the |{pgfmetadecoration}| environment.
\end{plainenvironment}

\begin{contextenvironment}{{pgfmetadecoration}\marg{name}}
  The Con\TeX t version of the |{pgfmetadecoration}| environment.
\end{contextenvironment}
